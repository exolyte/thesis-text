\chapter{Introduction}
\label{cha:intro}
For many years computer programs have been plagued by vulnerabilities that can be exploited in order to gain control over the execution of a program.
Several classes of these vulnerabilities arise from bugs in the source code and are related to low level control over memory.
Spatial memory issues are vulnerabilities that arise from accessing memory that is outside of the bounds that are supposed to be accessed.
Temporal memory issues on the other hand arise from accessing memory after the reference to that memory is supposed to have been invalidated.
Some famous examples of spatial and temporal memory issues are buffer overflows and use after free whereby an attacker can write to memory that is not supposed to be written to.

Over the past decades, several solutions to these issues have been presented and implemented.
One of the most popular solution is the usage of memory safe programming languages that deny the programmer low level access to the memory.
Instead, the memory is managed by a language runtime and deallocated by a garbage collector.
This solution, however, has a significant performance impact and is thus not applicable to performance sensitive programs such as kernels, browsers and low level system libraries.
These types of programs remain vulnerable and are often a vector for malware to gain control over a system.
A class of devices that are especially vulnerable to these issues are embedded devices since they have constrained resources and programmers require low level memory management features to gain reasonable performance.
With the advent of the Internet of Things (IoT), these embedded devices are expected to increase in number dramatically.

Over the past few years programming languages with strong type systems have gained renewed attention as a way to mitigate these issues and improve security.
One of the most popular of these is the Rust programming language.
Rust offers a number of security features, but in this thesis we will focus on Rust's ownership and borrowing system.
This system limits what a programmer can do with references to an object in memory, ensuring that memory is always accessed in a safe manner.
One of the most important principles of this system is the prevention of concurrent write access or concurrent read and write access to a region of memory as this can lead to data races.

However, when a Rust program is compiled, it is usually converted into an unsafe assembly language that does not have any of the safety guarantees.
This might, for example, open the programs up to attacks on the assembly level if they are linked with libraries that are not written in a programming language with the same safety guarantees.
In this thesis we explore an extension to the capability machine architecture CHERI in order to provide support on the assembly level for the safety guarantees that the Rust programming language offers.

Hardware capabilities are unforgeable pointers that represent authority over a region of memory.
They are comparable to software fat pointers in that they enforce bounds checking and hold permissions that specify how a region of memory can be used.
This model is very effective at combating spatial memory issues, but is less suited to enforce the temporal memory safety guarantees offered by languages with strong type systems.
Nevertheless, CHERI does offer some hardware primitives that software can build upon to strengthen their defense against temporal memory issues.
In this thesis we will add a new type of capability called \textit{borrowed capability}.
This new type of capability is designed to mimic Rust's ownership and borrowing system, but it could be seen in a wider context as an addition to the types of revocation that are already offered by CHERI.

In summary, the goal of this thesis is to design an extension to the CHERI architecture that offers the same safety guarantees as the Rust programming language.
To test our design, we will implement it in the Sail specification language, extend the LLVM project to create an assembler for our extension and run test programs on the emulator generated by Sail.
We start off by giving background information about Rust and CHERI as well as information about RISC-V and the Sail language which will be essential for the implementation of our design.
Next, we give an overview of our design, followed by the details of its implementation in Sail and the details of the LLVM extension.
Finally, we will evaluate our design by constructing a number of assembly example programs and reason about how well they mimic the Rust ownership and borrowing system.

%%% Local Variables: 
%%% mode: latex
%%% TeX-master: "thesis"
%%% End: 
