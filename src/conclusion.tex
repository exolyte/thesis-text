\chapter{Conclusion}
\label{cha:conclusion}
In this thesis we presented \textit{borrowed capabilities}, a new type of capability for CHERI that is designed to mirror the semantics of the Rust programming language.
We introduced the design for borrowed capabilities, revolving around binding capabilities to lifetime tokens and making these bound capabilities only dereferenceable when that lifetime token is present.
The design provides support for the ``aliasing XOR mutation'' principle by locking away capabilities in a borrow table and having both mutable and immutable borrow operations.
Concurrent use cases were taken into account by allowing lifetime tokens to split into multiple lifetime fractions that can be used by multiple threads at the same time.
We made a Sail implementation of both linear capabilities and borrowed capabilities and discussed the various design decisions relating to this implementation.
We extended the LLVM compiler project to be able to assemble programs using the instructions that were introduced in the design.
Finally we wrote sample assembly programs that mirror specific Rust programs to demonstrate how borrowed capabilities offer semantics that are similar to Rust's ownership and borrowing system and reasoned about the costs of the design of borrowed capbilities.
The assembly programs can be assembled by the LLVM assembler and run in the emulator that is produced by Sail.
From the assembly programs we can conclude that borrowed capabilities do indeed have semantics and guarantees that are similar to those that Rust provides.

There are some outstanding problems with borrowed capabilities that prevent them from being used as the target of a secure Rust compiler such as the behavior related to nested references.
Nevertheless, we think borrowed capabilities are a significant step towards supporting Rust's ownership and borrowing system at the assembly level.
While fitting the design of borrowed capabilities into CHERI is fairly simple and does not require expensive changes to CHERI, actually implementing borrowed capabilities in hardware could have significant extra costs.
The main limitations of borrowed capabilities such as lifetime id exhaustion may be mitigated or worked around.

We also considered borrowed capabilities as a new form of revocation in CHERI, supporting a model of mutual distrust between two parties.
While this thesis did not focus on the use of borrowed capabilities as a general form of revocation, we do think they are promising in this regard and this is an interesting direction to explore in future work.


%%% Local Variables: 
%%% mode: latex
%%% TeX-master: "thesis"
%%% End: 
