\section{CHERI}
Capability Hardware Enhanced RISC Instructions (CHERI) is an architecture neutral model of hardware supported capability machines, largely developed by researchers at Cambridge University.\cite{UCAM-CL-TR-951} This chapter starts off with a general introduction to capabilities which is followed by a deeper look into the specifics of the CHERI protection model and its implementation in the RISC-V ISA.

\subsection{Capabilities}
\label{sec:capintro}
Capabilities are unforgeable tokens that give a process access to some resource. They are created by a privileged entity such as the hardware or the kernel. This privileged entity also checks and enforces the correct use of capabilities. In the context of memory they are a method of addressing memory that is different from the traditional approach of using raw integer pointers. Owning a memory capability gives a process the rights to read from, write to or execute the contents of a location in memory, depending on the access permissions of the capability. Memory capabilities can usually give access to a range of memory locations which allows them to properly express concepts like arrays and structs. Figure \ref{fig:capability} depicts a capability layout of an early iteration of the CHERI design and allows for a simple explanation of how capabilities work.

\begin{figure}[h]
\centering
\definecolor{lightgray}{gray}{0.8}
\begin{bytefield}[endianness=big, bitwidth=.55em]{64}
    \bitheader{0,63} \\
    \bitbox{8}{\color{lightgray}\rule{\width}{\height}} & \bitbox{24}{\textbf{otype} (24 bits)} & \bitbox{31}{\textbf{permissions} (31 bits)} & \bitbox{1}{\textbf{s}} \\
    \bitbox{64}{\textbf{offset} (64 bits)} \\
    \bitbox{64}{\textbf{base} (64 bits)} \\
    \bitbox{64}{\textbf{length} (64 bits)}
\end{bytefield}
\caption{An old version of the CHERI capability layout.\cite{Watson2015CHERIAH}}
\label{fig:capability}
\end{figure}

A process that holds a capability like in figure \ref{fig:capability} is able to access the memory locations between \textit{base} and \textit{base+length} by modifying the \textit{offset} so that \textit{base+offset} points to the desired memory address. The process can freely modify the \textit{offset} field, but it is only allowed to alter the \textit{base} or \textit{length} fields in such a way that they point to a subset of the memory that the original capability pointed to. If the process tries to dereference the capability when the \textit{offset} field is set in such a way that it does not point to a memory location within the permissible range, some sort of error will be raised, such as a hardware exception in the case of CHERI. This prevents a whole range of possible security issues that are present in systems using raw pointers such as faulty pointer arithmetic and buffer overflows. The \textit{permissions} field contains a bitmap of various permissions such as read, write and execute. Modifying the \textit{permissions} field works similar to modifying the \textit{base} and \textit{length} fields in the sense that the process can only modify them in such a way that results in less permissions than the original capability. This form of monotonicity where a process can only reduce the rights on the capability it holds is central to the CHERI design. It gives a process a method to efficiently restrict the rights on the capabilities it delegates to components it may not trust while still retaining those rights for itself. The \textit{otype} and \textit{sealed} (\textit{s}) fields will be explained in section \ref{sec:sealed}.

\subsection{CHERI Capability Layout}
Figure \ref{fig:cheri_capability} shows the layout of 128-bit capabilities in the latest iteration of the CHERI specification. This capability format is intended for use in 64-bit architectures. While a 64-bit capability format intended for 32-bit architectures exists, this format is too limited to support the work in thesis so we will not consider it here.

\begin{figure}[h]
\centering
\definecolor{lightgray}{gray}{0.8}
\begin{bytefield}[endianness=big, bitwidth=.55em]{64}
    \bitheader{0,63} \\
    \bitbox{14}{\textit{p}'16} & \bitbox{3}{\color{lightgray}\rule{\width}{\height}} & \bitbox{16}{otype'18} & \bitbox{3}{\textit{$I_E$}} & \bitbox{8}{\textit{T}[11:3]} & \bitbox{5}{\textit{$T_E$}'3} & \bitbox{10}{\textit{B}[13:3]} & \bitbox{5}{\textit{$B_E$}'3} \\
    \bitbox{64}{\textit{a}'64}
\end{bytefield}
\caption{The current iteration of the CHERI capability layout.\cite{UCAM-CL-TR-951}}
\label{fig:cheri_capability}
\end{figure}

These capabilities consist of two 64-bit words with a number of fields:

\begin{itemize}
    \item p: The permissions field is a bitvector containing various permissions. It consists of two subfields:
        \begin{itemize}
            \item hperms: The hardware permissions field is 12 bits wide and contains the permissions defined by CHERI such as read, write, execute and various others.
            \item uperms: The user permissions field is 4 bits wide and is reserved for application specific uses.
        \end{itemize}
    \item reserved: There are 3 unused bits reserved for possible future or implementation specific use.
    \item otype: The object type field is used by sealed capabilities as explained in section \ref{sec:sealed}.
    \item bounds: The next 27 bits are used to define the bounds of the capability. This is a compressed representation of the \textit{base} and \textit{length} fields described in section \ref{sec:capintro}. The bounds bits are explained in more detail in section \ref{sec:bounds}.
        \begin{itemize}
            \item $I_E$: the internal exponent.
            \item E: the exponent.
            \item T: the top.
            \item B: the base.
        \end{itemize}
    \item a: The address field holds the referenced memory location. This field is similar to a raw integer pointer.
\end{itemize}

\subsection{Capability Tags}
In CHERI, each register and each capability-aligned memory location is associated with a 1-bit tag that is kept out of band. This means the tag cannot be directly accessed by the software. A tag indicates the presence of a valid capability and is read from and written to in certain hardware operations. The tag allows the hardware to guarantee that the software does not alter its capabilities in a way that is not allowed. In general, if a program wants to alter a capability in any way, it needs to do this through CHERI capability manipulation instructions. If the program writes to a memory location or register where a capability resides using traditional write instructions, the hardware will clear the tag which invalidates the associated capability. It is then no longer considered a valid capability and it cannot be used anymore by the majority of CHERI instructions. It is the responsibility of the compiler to ensure that no necessary capabilities are accidentally invalidated. 

\subsection{PCC \& DDC}
Two of the most important special registers added by CHERI are the program counter capability (PCC) and default data capability (DDC). The PCC extends the traditional program counter as a capability which allows applications to constrain execution of code to memory locations for which the application holds a capability with execute permissions. The DDC register holds a capability through which traditional, non-capability memory accesses are indirected. This is a hybridization feature that allows for running non-CHERI compiled binaries on a CHERI CPU with some of the CHERI safety guarantees. Every legacy load or store instruction is checked against the bounds and permissions of the capability in the DDC register. If the memory access is allowed by the capability in the DDC register, the instruction completes successfully. If however the memory access is not allowed, the instruction that is attempting to access memory will trap.

\subsection{Capability Permissions}
The permissions field is a simple bitvector where each bit corresponds to a certain permission. An application can manipulate permissions on their capabilities through the \textit{CAndPerm} instruction. The \textit{CAndPerm} instruction takes a capability and a bitmask as arguments and outputs a capability with that bitmask applied to the permissions field. This operation ensures that an application cannot give itself more permissions since a bitmask can not be used to set a bit that was not originally set. This means that the capability rights monotonicity is preserved.
The hardware permissions (\textit{hperms}) subfield is a 12-bit field that contains the permissions that are interpreted by the hardware. Given below is the interpretation of each bit in the vector, starting from the least significant bit. The \textit{Global} and \textit{Permit\_Store\_Capability} permissions are further explained in section \ref{sec:global}. The \textit{Permit\_Seal}, \textit{Permit\_CInvoke} and \textit{Permit\_Unseal} permissions are further explained in section \ref{sec:sealed}.
\begin{itemize}
    \item Global: Allow this capability to be stored by all capabilities that have the \textit{Permit\_Store\_Capability} permission.
    \item Permit\_Execute: Allow this capability to be used to fetch instructions.
    \item Permit\_Load: Allow this capability to be used to load data from memory.
    \item Permit\_Store: Allow this capability to be used to store data in memory.
    \item Permit\_Load\_Capability: Allow this capability to be used to load capabilities from memory. The \textit{Permit\_Load} bit is also required.
    \item Permit\_Store\_Capability: Allow this capability to be used to store capabilities in memory. The \textit{Permit\_Store} bit is also required.
    \item Permit\_Store\_Local\_Capability: Allow this capability to be used to store capabilities that do not have the \textit{Global} bit set. The \textit{Permit\_Store} and \textit{Permit\_Store\_Capability} are also required.
    \item Permit\_Seal: Allow this capability to be used to seal other capabilities.
    \item Permit\_CInvoke: Allow this capability to be used as an argument to CInvoke.
    \item Permit\_Unseal: Allow this capability to be used to unseal other capabilities.
    \item Permit\_Set\_CID: Allow this capability to be used to set the architectural compartment ID.
    \item Access\_System\_Registers: Allow this capability to be used to access privileged registers. The interpretation of this bit is architecture-specific.
\end{itemize}
The 4-bit user permissions field (\textit{uperms}) is intended for application specific use and is not interpreted by the hardware. These bits follow the same principles as the \textit{hperms} bits, but an application can give its own semantic meaning to them.

\subsection{Global and Local Capabilities}
\label{sec:global}
A capability can either be \textit{global}, which means it can be stored in memory by any capability that has the \textit{Permit\_Store\_Capability} permission, or \textit{local}, which means it can only be stored by capabilities that have the \textit{Permit\_Store\_Local\_Capability} permission. These low level permissions can be used to construct higher level restrictions on the propagation of capabilities depending on which capabilities have the \textit{Global} and \textit{Permit\_Store\_Local\_Capability} bits set. This propagation restriction depends on different components of software having access to different parts of memory and having access to capabilities with differing \textit{Permit\_Store\_Local\_Capability} permissions. For example, if a component has exclusive access to some piece of memory (i.e. no other component has a capability pointing to that part of memory) through a capability with \textit{Permit\_Store\_Local\_Capability} and it has no other capabilities with \textit{Permit\_Store\_Local\_Capability}, it means that \textit{local} capabilities cannot be passed on to other components. As another example, consider a component that has a capability pointing to a shared part of memory with \textit{Permit\_Store\_Local\_Capability}. This means that the component can store \textit{local} capabilities to that shared part of memory. Other components can then load the \textit{local} capability from shared memory, but they cannot store it again, unless they have a capability with \textit{Permit\_Store\_Local\_Capability} themselves.
Some possible applications for this system are restricting local stack capabilities to be stored only on the local stack or letting capabilities flow in only one way through a shared buffer.\cite{UCAM-CL-TR-951} %page 59

\subsection{Linear Capabilities}
Linear capabilities is a concept that has been proposed by researchers and is currently being considered for adoption in CHERI. We briefly introduce it here because the work in this thesis makes extensive use of linear capabilities. Linear capabilities are capabilities that can not be copied in any way. Linear capabilities can be used to enforce well-bracketed control flow, local state encapsulation \cite{10.1145/3290332} or to represent memory resources described by spatial separation logic predicates.\cite{10.1145/3341688} In this thesis we will find additional uses for linear capabilities.

\subsection{Sealed Capabilities}
\label{sec:sealed}
In their most basic form, sealed capabilities are capabilities that cannot be modified or dereferenced. Normal unsealed capabilities are identified by an otype value of $2^{18} - 1$, for all other values of the 18-bit \textit{otype} field, the capability is considered sealed. CHERI currently reserves 16 otype values for specific interpretations, all other otype values can be used for code-data pairs, further explained below. The reserved otype values range from $2^{18} - 1$ to $2^{18} - 16$.The first of these values is used for unsealed capabilities, as mentioned above. The second one is used for sealed entry (sentry) capabilities, the other values in the range have currently not been assigned. Sentry capabilities are intended to be used as pointers to code that can be executed and jumped to by control flow instructions. The modification restriction of sealed capabilities fits this use case well because it prevents applications from creating unintended control flows.
The main use case for sealed capabilities and the \textit{otype} field is the concept of code-data pairs. Code-data pairs are designed to support compartmentalization of software within a single address space. As their name implies, a code-data pair is a pair of two capabilities, one of them pointing to the code of a compartment and the other pointing to the data used by this compartment. Code-data pairs are created by sealing both capabilities with the CSeal instruction. This instruction takes two input operands. The first one is the code or data capability to be sealed and needs to have the \textit{Permit\_CInvoke} permission. The second one is a capability with the \textit{Permit\_Seal} permission. CSeal sets the otype field on the code or data capability to the value in the address field of the second operand. Two capabilities with the same otype value are considered code-data pairs and can be used as input operands of the CInvoke instruction. The CInvoke instruction unseals the code-data pair, moves the code capability to the PCC and places the data capability in a set register. This system can be used as a security domain transition between two mutually distrusting compartments. The caller cannot tamper with the callee since the references it holds to the callee --- the code-data pair --- are sealed and the callee cannot access any of its caller's state, except for the arguments that the caller passed through the CPU registers.

\subsection{Capability Bounds}
\label{sec:bounds}
While the full calculations of the bounds of a capability are outside of the scope of this thesis, we will explain the general principle behind them in this section. The $I_E$ field is a bit to indicate whether the \textit{E} field is used. When $I_E$ is equal to zero, the size of the \textit{E} field is 0 and $T_E$ and $B_E$ are part of the \textit{T} and \textit{B} fields respectively. If $I_E$ is equal to one, $T_E$ and $B_E$ are joined together to form the \textit{E} field with a size of 6 bits, shortening the \textit{T} and \textit{B} fields in return. These bits that are in this case missing from \textit{T} and \textit{B} are considered to be zeros. The \textit{B} and \textit{T} fields are used in the calculation of the base and top values --- the addresses which \textit{a} must lie between. The $50 - E$ most significant bits of the base and top are copied from the \textit{a} value, possibly incremented or decremented with a correction value. This correction value is calculated from various comparisons between certain bits of the \textit{T}, \textit{B} and \textit{a} fields. The next 14 bits of the base value are directly copied from the \textit{B} field. The next 14 bits of the top value are copied from the \textit{T} field, extended with 2 bits, based on the \textit{B} and $I_E$ field. The remaining \textit{E} bits are zeros. This means that using the \textit{E} field by setting the $I_E$ bit increases the representable range to the entire address space at the cost of accuracy, introducing alignment requirements. The accuracy cost is inversely proportional to the size of the \textit{E} value.

\subsection{Capability Exceptions}
In general, when an application does something that is not allowed, such as dereferencing a memory location without the proper permissions or modifying a sealed capability, the hardware will raise an exception and will transfer execution to the appropriate exception handler. This allows an operating system to handle the exception in a correct manner.

\subsection{RISC-V Implementation}
While an implementation of CHERI in MIPS exists, in this thesis we work with the RISC-V implementation of CHERI so we will restrict this section to RISC-V and to the parts of CHERI-RISC-V that are relevant to this thesis. CHERI-RISC-V is an extension of the RISC-V instruction set that implements the CHERI protection model. CHERI-RISC-V uses instructions space that is reserved for extensions in the RISC-V instruction set. This instruction space differs depending on whether compressed RISC-V instructions are used. CHERI-RISC-V support both compressed and non-compressed instructions, but makes some changes for compressed instructions to account for the reduced opcode space.
CHERI-RISC-V supports both a split and a merged register file. With a split register file, registers able to hold capabilities are separate from the integer registers. In contrast, with a merged register file all registers are able to hold both capabilities and integers where the upper 64 bits of a register are ignored if an operand register is interpreted as an integer.
Capabilities in CHERI-RISC-V have an extra 1-bit field used for the encoding-mode flag. The addition of this flag reduces the amount of reserved bits from three to two. The encoding-mode flag is used on capabilities that are used as the PCC. When the bit is off, the system is in integer encoding mode, meaning that operands to traditional RISC-V loads and store instructions are interpreted as traditional integer addresses. When the bit is set, the system is in capability encoding mode, meaning that operands to traditional RISC-V load and store instructions are interpreted as capabilities. The big advantage of this mode is that traditional RISC-V load and store instructions can be used with capabilities, saving a significant amount of opcode space.
