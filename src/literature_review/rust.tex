\section{The Rust programming language}
\label{sec:rustbackground}
Rust is a relatively new programming language that aims to provide the high performance associated with other low-level programming languages such as C and C++ while simultaneously preventing a number of classes of bugs related to memory and thread safety that typically arise due to the low-level resource management features necessary to achieve such high performance.
Rust accomplishes this at compile time by statically checking if the source code satisfies the rules of Rust's strong ownership-based type system.
Because these rules are conservative and might reject correct programs, Rust provides an \textit{unsafe} mode in which some of these rules do not need to be satisfied.
Since this thesis is about the safety Rust provides, we only consider \textit{safe} Rust in this section.

\subsection{Ownership}
Central to Rust's resource management model is the concept of ownership.
When a value gets created, ownership of that value will be assigned to a variable.
This value stays valid until its owner goes out of scope after which it cannot be used anymore.
Ownership of a value can be transferred to another variable, when this happens the value cannot be accessed anymore through the original variable.
This is illustrated in listing \ref{code:move} which fails to compile because \textit{a} gets accessed after its value has been moved to \textit{b}.
\begin{lstlisting}[language=Rust,frame=single,caption=Moving a variable,label=code:move]
fn main() {
  let a = String::from("This is a heap allocated string");
  let b = a;
  println!("{}", a);
}
\end{lstlisting}

\subsection{Borrowing}
\label{subsec:borrowing}
As mentioned, Rust follows the principle of ``aliasing XOR mutation'' to prevent a number of bugs related to memory safety and data races.
The borrowing system has a set of rules that enforces this principle.
A value can be borrowed by variables other than its owner, giving those variables temporarily the right to use the value.
There are two types of borrows: a mutable borrow and an immutable (or shared) borrow. 

A mutable borrow allows the borrowing variable to both read and mutate the contents of the value.
To prevent multiple variables mutating the same value at the same time and to prevent variables from reading a value that is being altered, the owner temporarily loses its right to access or lend the value while a mutable borrow is in scope.
As a result, only the borrowing variable is allowed to access the value while a mutable borrow exists.

A shared borrow only allows a borrowing variable to read the contents of the value.
While a shared borrow is in scope, the owner cannot mutate or mutably lend the value anymore.
It is however allowed to read the value and immutably lend the value again.
This means that an unlimited amount of shared borrows can read a value at the same time, but it cannot be mutated while any of those shared borrows exist.

Listing \ref{code:borrowexample} shows a simple program that creates a mutable variable \textit{x}, borrows it mutably and increments the value through that borrow 
Then, after the mutable borrow has ended, the program borrows \textit{x} twice, but this time immutably and uses those borrows to print the value. Note that during the immutable borrow, the owner variable \textit{x} stays available to read from.

\begin{lstlisting}[language=Rust,frame=single,caption=Borrowing an integer,label=code:borrowexample]
fn main() {
  let mut x = 5;
  {
    let m = &mut x;
    *m += 1;
  }
  {
    let s1 = &x;
    let s2 = &x;
    println!("{}, {}, {}",x, s1, s2);
  }
}
\end{lstlisting}

\subsection{Lifetimes}
Section \ref{subsec:borrowing} explained the rules around borrowing, but did not explain how a borrow actually ends.
For this, Rust has a concept called lifetimes.
Every borrow has a lifetime which determines when the borrow is valid.
In order to lighten the load on the programmer, the Rust compiler usually implicitly infers the lifetimes of borrows.
However, in some situations it is necessary to explicitly annotate the Rust code with lifetimes.
Listing \ref{code:lifetime_semantics} shows such a situation.
In this example, the function \textit{longest} takes two string references as arguments and returns a reference to one of those.
The function does not know which arguments it will be called with and thus cannot implicitly infer any information about their lifetimes.
This also means it cannot give any guarantees about the lifetime of its return value.
This is why it is necessary to link the arguments to the return value using lifetime parameter annotations.
These annotations allow the function to tell its caller that the returned reference has the same lifetime as the arguments that were passed in.
\begin{lstlisting}[language=Rust,frame=single,caption=Lifetime Example,label=code:lifetime_semantics]
fn main() {
  let string1 = String::from("abcd");
  let string2 = "xyz";

  let result = longest(string1.as_str(), string2);
  println!("The longest string is {}", result);
}

fn longest<'a>(x: &'a str, y: &'a str) -> &'a str {
  if x.len() > y.len() {
    x
  } else {
    y
  }
}
\end{lstlisting} %TODO reference Rust book (this is copied)
\label{fig:lifetime_semantics}

\subsection{Borrowing Specifics}
Borrowing has a number of specific interactions depending on the situation and the data types that get borrowed.
Because these interactions are important when we model the Rust semantics on the hardware level, we introduce them in more depth here.

\subsubsection{Reborrowing}
\label{sec:backgroundreborrow}
In section \ref{subsec:borrowing} we stated that a value can not be immutably borrowed while it is mutably borrowed.
This is not entirely accurate as listing \ref{code:reborrow_semantics} shows.
In the example we create a mutable value and borrow it mutably.
We then create a shared borrow from the mutable borrow.
This phenomenon where a borrow gets borrowed again is called a reborrow.
Reborrowing a borrow works similarly to borrowing an owner in the sense that the original borrow --- which we will also call the \textit{parent} --- loses (some of) its permissions while the reborrowed reference --- which we will also call the \textit{child} --- is active.
This can be seen in the difference between listing \ref{code:reborrow_semantics} and listing \ref{code:reborrow_semantics_wrong}.
In listing \ref{code:reborrow_semantics}, the mutable borrow is not used until the lifetime of the shared borrow has already ended and the code compiles correctly.
However, if we move the increment to the value up as in Listing \ref{code:reborrow_semantics_wrong}, the Rust compiler returns an error because the shared borrow still gets used after the increment.
This reborrow feature adds a lot of expressivity to Rust's fairly strict semantics and will have a major impact on our design choices for borrowed capabilities.

\noindent
\begin{tabular}{p{6.65cm} p{6.65cm}}
    \begin{lstlisting}[language=Rust,frame=single,caption=Reborrow Example,label=code:reborrow_semantics]
fn main() {
  let mut x = 5;
  {
    let m = &mut x;
    {
      let s = &(*m);
      println!("{}", s);
    }
    *m += 1;
  }
}
    \end{lstlisting}

    &

    \begin{lstlisting}[language=Rust,frame=single,caption=Wrong Example,label=code:reborrow_semantics_wrong]
fn main() {
  let mut x = 5;
  {
    let m = &mut x;
    {
      let s = &(*m);
      *m += 1;
      println!("{}", s);
    }
  }
}
    \end{lstlisting}
\end{tabular}

\subsubsection{Compound Data Types}
While safe Rust does have a large amount of limitations around slices of data types such as arrays, the borrowing of individual fields of structs is well supported.
This means that it is possible to have multiple mutable references or both mutable and shared references to the same struct at the same as long as they point to different members of the struct.
Listing \ref{code:struct_semantics} shows an example where both fields of a struct get mutably borrowed and independently altered.
This code works and is considered safe Rust.

\begin{lstlisting}[language=Rust,frame=single,caption=Borrowing struct fields,label=code:struct_semantics]
struct S {
  a: i32,
  b: i32
}

fn main() {
  let mut s = S{a: 0, b: 0};
  {
    let a = &mut s.a;
    let b = &mut s.b;
    *a += 1;
    *b += 2;
    println!("{}, {}", a, b);
  }
}
\end{lstlisting}

\subsubsection{Nested References}
Rust has some complex rules surrounding the loading of references through a borrow that points to a data structure that contains so-called ``nested references''.
In this section we will explain some of these peculiarities through some simple examples.

The first rule pertains to the loading of references through a borrow in general.
Listing \ref{code:nested_move} shows an instantiation \textit{s} of a struct that holds a reference to an integer.
Then, \textit{s} gets borrowed by the variable \textit{b} and the variable \textit{n} tries to load the reference to the integer.
This, however, fails because this code tries to move the reference out of the struct into \textit{n}.
The simple solution to this issue is to borrow the reference to the integer instead, as shown in listing \ref{code:nested_borrow}.
While this may seem like a trivial difference, it is important to note because it means that ownership of a reference cannot be moved out of a struct via a borrow and it means that the variables that have loaded such a reference are bound to a lifetime themselves.
\begin{tabular}{p{6.65cm} p{6.65cm}}
    \begin{lstlisting}[language=Rust,frame=single,caption=Move reference,label=code:nested_move]
struct S<'a> {
  n: &'a mut i32
}

fn main() {
  let mut x= 0;
  let mut s = S{n: &mut x};
  let b = &mut s;
  let n = b.n;
}
    \end{lstlisting}

    &

    \begin{lstlisting}[language=Rust,frame=single,caption=Borrow reference,label=code:nested_borrow]
struct S<'a> {
  n: &'a mut i32
}

fn main() {
  let mut x= 0;
  let mut s = S{n: &mut x};
  let b = &mut s;
  let n = &b.n;
}
    \end{lstlisting}
\end{tabular}

In the next example in listing \ref{code:nested_lifetime} we show some code where the lifetime of this loaded reference is relevant.
In this code we pass the borrow for the struct into the \textit{nested\_borrow} function where it is reborrowed by the \textit{rb} variable.
We then load the integer reference to the variable \textit{n} and try to return \textit{n}.
This fails to compile because the lifetime of \textit{n} is bound to the lifetime of \textit{rb} which does not outlast the function.

\begin{lstlisting}[language=Rust,frame=single,caption=Reborrow Example,label=code:nested_lifetime]
struct S<'a> {
  n: &'a mut i32
}

fn main() {
  let mut x= 0;
  let mut s = S{n: &mut x};
  let b = &mut s;

  let n = nested_borrow(b);
  println!("{}", n);
}

fn nested_borrow<'a>(mut b: &'a mut S) -> &'a i32 {
  *b.n += 1;
  let rb = &mut b;
  let n = &rb.n;
  return n;
}
\end{lstlisting}

Our last example in listing \ref{code:nested_immutmut} shows that \textit{s} is immutably borrowed by \textit{b} and \textit{n} then tries to get a mutable reference through \textit{b}.
This operation, however, fails to compile because Rust does not allow the loading of mutable references through shared references.
The reason for this is that there could be multiple shared references that could be used to load the mutable reference and this behavior would violate the ``aliasing XOR mutation'' principle.
\begin{lstlisting}[language=Rust,frame=single,caption=Mutable reference through shared borrow,label=code:nested_immutmut]
struct S<'a> {
  n: &'a mut i32
}

fn main() {
  let mut x= 0;
  let mut s = S{n: &mut x};
    
  let b = &s;
  let n = &mut b.n;
}
\end{lstlisting}

We will not support all of these rules in our design for borrowed capabilities.

\subsection{Rust Borrow Invariants}
\label{sec:obsinvariants}
In this section we will give a more in-depth explanation of what the Rust safety guarantees are in the form of a set of invariants.
Kan et al. \cite{Kan2018AnEO} identify a set of five invariants that must hold when borrowing in Rust.
These invariants that they refer to as the ownership and borrowing system (OBS) invariants are listed in figure \ref{fig:obsinvariants}.
The OBS invariants ensure that each reference points to a valid block of memory and that each valid block can either be accessed through multiple immutable references or one mutable reference.

\begin{figure}[h]
\centering
\begin{enumerate}
    \item \textit{Unique owner invariant}: Each block has a unique owner.
    \item \textit{Lifetime inclusion invariant}: If $x$ borrows or reborrows $y$ then the lifetime of $x$ should always be within the lifetime of $y$ in order to avoid dangling pointers.
    \item \textit{Lifetime disjoint invariant}: There are \textit{no} two references to the same referent such that their lifetimes intersect and one of them is a mutable reference.
    \item \textit{Writing permission invariant}: If $x$ borrows or reborrows $y$ immutably then the writing permission of $y$ should be disabled until the end of $x$'s lifetime.
    \item \textit{Reading and writing permissions invariant}: If $x$ borrows or reborrows $y$ mutably then both the reading and writing permissions of $y$ should be disabled until the end of $x$'s lifetime.
\end{enumerate}
\caption{The five OBS invariants.\cite{Kan2018AnEO}}
\label{fig:obsinvariants}
\end{figure}

\subsection{Rust Static Guarantees}
Because the compiler automatically deallocates values after their owner goes out of scope, the programmer cannot make typical deallocation related mistakes such as \textit{double free} or \textit{invalid free}.
An owner can only go out of scope when no more borrows exist, meaning that a \textit{use after free} bug is impossible.
The borrow system eliminates the necessity of using raw pointers and prevents problems relating to null pointers or buggy pointer arithmetic.
Rust also prevents simultaneous read and write access to a value, making data races impossible.
All of these static safety guarantees also result in Rust's ability to prevent thread related bugs and enforce thread safety.

%TODO ref rustbelt and rust book
