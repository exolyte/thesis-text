\chapter{Evaluation}

\section{Assembly Explanation}
In this section we explain the assembly code that is required to set up the program and is common to all of the examples we will show in this chapter.
To save space and to focus on the essential part of the assembly, we will not repeat the code that is given here.

The assembly snippet starts off by defining the \textit{\_start} label which is read by the emulator and used to determine where the emulator should start executing code.
The code at the \textit{\_start} label loads the address of the \textit{\_trap\_vector} label in the \textit{t0} register and then sets that address as the address to trap to when an instruction encounters an exception with the \textit{csrw} instruction.
The following two lines do something similar but set the return address that is returned to when executing the \textit{mret} instruction instead.
Then the \textit{mret} causes the emulator to enter user mode and jump to the \textit{\_user} label.
The \textit{\_user} label is where the code that we give throughout this chapter will be placed when executing the examples.
The code at the \textit{\_trap\_vector} label loads the value 3 in a register that will be used to pass the return code of the program to the emulator and then jumps to the \textit{\_exit} label.
The code at the \textit{\_exit} label writes the value in \textit{gp} to the \textit{tohost} address which is considered communication with the emulator.
The emulator considers any value with the least significant bit set to 1 as a message to stop the emulation with a return code defined by the number in the other bits of the value.
A return code of zero means that the program exited successfully while any other value signifies an error code that corresponds to a failure in the program.
In the section of code at \textit{\_trap\_vector} we set the value to 3 which corresponds to a failure with error code 1.
This happens when some instruction in our program raised an exception.
In the other assembly snippets we will give throughout this chapter, we will set the value of \textit{gp} to 1 to signify success.
The final line of the assembly defines the \textit{tohost} symbol and address so that the emulator knows which address it should be listening to.
\begin{verbatim}
        .global _start
_start:
        la t0,  _trap_vector
        csrw	mtvec,t0

        la      t0, _user
        csrw    mepc, t0
        mret

.align 4
_trap_vector:
        li      gp,3
        j       _exit

_exit:
        auipc   t5, 0x1
        sw      gp, tohost, t5
        j       _exit

_user:
        ...

.align 6; .global tohost; tohost: .dword 0;
\end{verbatim}
%TODO refer to the github where this was inspired from

\section{Instruction Testing}

\section{Comparison to Rust}

\section{Performance}
