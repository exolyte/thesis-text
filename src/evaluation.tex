\chapter{Evaluation}

\section{Assembly Explanation}
In this section we explain the assembly code that is required to set up the program and is common to all of the examples we will show in this chapter.
To save space and to focus on the essential part of the assembly, we will not repeat the code that is given here.

The assembly snippet starts off by defining the \textit{\_start} label which is read by the emulator and used to determine where the emulator should start executing code.
The code at the \textit{\_start} label loads the address of the \textit{\_trap\_vector} label in the \textit{t0} register and then sets that address as the address to trap to when an instruction encounters an exception with the \textit{csrw} instruction.
The following two lines do something similar but set the return address that is returned to when executing the \textit{mret} instruction instead.
Then the \textit{mret} causes the emulator to enter user mode and jump to the \textit{\_user} label.
The \textit{\_user} label is where the code that we give throughout this chapter will be placed when executing the examples.
There is some common set-up code under the \textit{\_user} that is shown here and will be explained later.

The code at the \textit{\_trap\_vector} label loads the value 3 in a register that will be used to pass the return code of the program to the emulator and then jumps to the \textit{\_exit} label.
The code at the \textit{\_exit} label writes the value in \textit{gp} to the \textit{tohost} address which is considered communication with the emulator.
The emulator considers any value with the least significant bit set to 1 as a message to stop the emulation with a return code defined by the number in the other bits of the value.
A return code of zero means that the program exited successfully while any other value signifies an error code that corresponds to a failure in the program.
In the section of code at \textit{\_trap\_vector} we set the value to 3 which corresponds to a failure with error code 1.
This happens when some instruction in our program raises an exception.
At the end of the code under the \textit{\_user} label we set the value of \textit{gp} to 1 to signify successful completion of the code.

The rest of the code under the \textit{\_user} label consists of set-up code that we need for our example programs.
The \textit{CSpecialRW} instruction gets a capability with the maximum amount of permissions from the DDC register and stores it in the $c2$ register.
The binary value \textit{111101} is then loaded in $x1$, representing the presence of the \textit{Global} and the load and store CHERI permissions, but none of the others.
These permissions are applied to the capability in $c2$ through the \textit{CAndPerm} instruction.
We then set the offset of the capability to point to \textit{0x82000000} which is an arbitrary address in system memory.
The first \textit{CSetBounds} instruction duplicates the capability to $c20$, sets the base to \textit{0x82000000} and sets the length to 8 which is the length of two 32 bit words.
This capability in $c20$ will be used in one of the later program examples.
The second \textit{CSetBounds} instruction does something similar to the capability in $c2$, but sets the length to 4 instead.
The last two lines of the code under the \textit{\_user} label make both of these capabilities linear so that they behave like Rust owner variables.

The final line of the assembly defines the \textit{tohost} symbol and address so that the emulator knows which address it should be listening to.
\begin{verbatim}
        .global _start
_start:
        la      t0, _trap_vector
        csrw	mtvec, t0

        la      t0, _user
        csrw    mepc, t0
        mret

.align 4
_trap_vector:
        li      gp, 3
        j       _exit

_exit:
        auipc   t5, 0x1
        sw      gp tohost, t5
        j       _exit

_user:
        cspecialrw  c2, ddc, c0
        li          x1, 0b111101
        CAndPerm    c2, c2, x1
        li          x1, 0x82000000
        CIncOffset  c2, c2, x1
        li          x1, 8
        CSetBounds  c20, c2, x1
        li          x1, 4
        CSetBounds  c2, c2, x1
        CMakeLinear c2, c2
        CMakeLinear c20, c20
        
        ...
        
        li  gp,1
        j   _exit

.align 6; .global tohost; tohost: .dword 0;
\end{verbatim}
%TODO refer to the github where this was inspired from

\section{Comparison to Rust}
In this section we introduce assembly programs that mirror the functionality of the Rust examples that we discussed in section \ref{sec:rustbackground}.
\tomi{Make sure to place these examples somewhere in your repo, and clearly indicate in the readme and in the thesis where they are.}
\subsection{Borrowing}
We start off with the example that we gave to illustrate borrowing in listing \ref{code:borrowexample}.
The assembly equivalent of this Rust program is shown in figure \ref{fig:asmborrowexample}.
The first two lines load an integer to integer register $x1$ and store that integer in the memory pointed to by the capability in capability register $c2$.
This is somewhat similar to creating a new variable, $x$, and assigning it a value.

The next five lines start by creating a lifetime token, which is roughly similar to opening a new scope\tomi{true, but I would explicitly mention the difference, where in rust the lifetime is attached to the pointer itself, whereas here we really reified the scope}.
The lifetime token is then used to mutably borrow $x$ with the borrowed capability $m$ being placed in register $c2$ and the resulting index token being placed in $c3$.
The borrow operation also locks the capability corresponding to $x$ away in the borrow table which ensures that it cannot be used to read or write from, just as Rust guarantees.
The borrowed capability is then used to load the value, increment it and store it again.
These load and store operations succeed because the lifetime token is in the lifetime token register of $c31$.

In the following block the lifetime token is killed and then used with the index token in $c3$ to retrieve the original capability corresponding to $x$.
The capability is then moved back to $c2$ which results in the same state as the start of the program, with the exception of the incremented integer in $x1$ and memory.

In the following lines, the assembly program diverges slightly from the original Rust program. \tomi{add forward reference to discussion section here}
This is because borrowing $x$ results in the capability being locked away in the borrow table which makes it impossible to dereference while in the Rust example, the $x$ variable is read from within the lifetime of the borrows.
The solution is to create an immutable shell borrow of $x$ under its own lifetime that can then be copied and reborrowed which is done in lines 13-15.
While this is semantically different from the original Rust program, it matches the behavior for immutable borrows where the owner can be used to read from, but not write to the resource.
In line 17 a sublifetime is created which is then used to reborrow a copy of the $x_shell$\tomi{watch out for underscores (also further on)} variable in lines 18-20.
We will expand more on reborrows in the next section.
Because we don't care about retrieving the copy of the $x_shell$ variable, we set the output for the index token in the borrow operations to $c0$ which results in no capability being locked away in the borrow table and no index token being created.

The capabilities corresponding to $x_shell$, $s1$ and $s2$ can then be used to print the value they point to, which we represent in a simplified manner by loading the value they point to.
Because $x_shell$ was borrowed under a different lifetime than $s1$ and $s2$, the lifetime tokens must be moved around to allow for dereferencing the capability in $c2$.
The last four lines kill the sublifetime, use that dead lifetime token to remove the child from the parent lifetime which then enables the parent lifetime to be killed.
The dead parent lifetime token can then be used with the index token in $c5$ to retrieve the original mutable capability corresponding to the $x$ variable.

\begin{figure}[h]
\begin{lstlisting}[style=custASM, numbers = left ,xleftmargin=1.5em]
li             x1 5
sw.cap         x1 0(c2)    #let mut x = 5;

CCreateToken   c31 c0      #{
CBorrowMut     c3 c2 c31   #m = &mut x;
lw.cap         x1 0(c2)
addi           x1 x1 1
sw.cap         x1 0(c2)    #*m += 1;

CKillToken     c31 c31     #}
CRetrieveIndex c3 c3 c31
CMove          c2 c3

CCreateToken   c31 c0      #{
CBorrowImmut   c5, c2, c31 #let x_shell = &x;
CMove          c3, c2      #let x_temp = x_shell;
CMove          c30, c31
CCreateToken   c31, c30    #{
CBorrowImmut   c0, c3, c31 #let s1 = &x_temp;
CMove          c4, c2      #let x_temp = x_shell;
CBorrowImmut   c0, c4, c31 #let s2 = &x_temp;

lw.cap         x1, 0(c3)   #println!("{}", s1);
lw.cap         x1, 0(c4)   #println!("{}", s2);
CMove          c29, c31
CMove          c31, c30
lw.cap         x1, 0(c2)   #println!("{}", x_shell);

CKillToken     c29, c29    #}
CUnlockToken   c31, c29
CKillToken     c31, c31    #}
CRetrieveIndex c5, c5, c31
\end{lstlisting}
\caption{An assembly program illustrating borrowing.}
\label{fig:asmborrowexample}
\end{figure}

\subsection{Reborrowing}
While we did have an example of reborrowing in the previous section, reborrowing also exists explicitly in Rust as seen in section \ref{sec:backgroundreborrow}.
In this section we give an assembly program in figure \ref{fig:asmreborrowexample} that corresponds to the Rust program shown in listing \ref{code:reborrow_semantics}.

The first four lines of the program are similar to the previous assembly program.
On line 6 we create a new lifetime token as the child of the token in $c30$ which can be used to reborrow borrowed capabilities with that lifetime.
This is what we do on line 7, where the mutable borrowed capability gets locked away in the borrow table, the new immutable borrowed capability takes its place and the index token gets placed in $c4$.
This means that an error such as the one in listing \ref{code:reborrow_semantics_wrong} where the mutable reference is used while the immutable reference exists, cannot occur.
The integer value is then loaded through the immutable borrowed capability.

The lifetime token for the immutable borrow is then killed and used to unlock the parent lifetime token.
This guarantees that the immutable borrow cannot be used anymore and it is safe to retrieve the mutable borrow.
This is done in line 12 through the index token in $c4$.

The rest of the program is once again similar to the program in the previous section.
\begin{figure}[h]
\begin{lstlisting}[style=custASM, numbers = left ,xleftmargin=1.5em]
li             x1 5
sw.cap         x1 0(c2)    #let mut x = 5;
CCreateToken   c30 c0      #{
CBorrowMut     c3 c2 c30   #m = &mut x;

CCreateToken   c31 c30     #{
CBorrowImmut   c4 c2 c31   #let s = &(*m);
lw.cap         x1 0(c2)    #println!("{}", s);

CKillToken     c31 c31     #}
CUnlockToken   c30 c31
CRetrieveIndex c4 c4, c31

CMove          c31 c30
lw.cap         x1 0(c4)
addi           x1 x1 1
sw.cap         x1 0(c4)    #*m += 1;

CKillToken     c31 c31     #}
CRetrieveIndex c3 c3 c31
\end{lstlisting}
\caption{An assembly program illustrating rereborrowing.}
\label{fig:asmreborrowexample}
\end{figure}

\subsection{Borrowing Struct Fields}
In this example we use the capability in $c20$ that points to two 32 bit words.
For simplicity we start by moving it to $c2$.
We then load 0 in the lower word corresponding to the struct's $a$ field, increase the offset on the capability to point to the upper word and load 0 in the upper word corresponding to the $b$ field.
To create borrows of the individual fields we first create a borrow of the full capability in $c2$ which we then split between the lower and upper word with the \textit{CSplitCap} instruction on line 10.
After this instruction the capability pointing to only the lower word, corresponding to $a$, is present in the $c3$ register while the capability pointing to only the upper word is in the $c2$ register.
We then increment the values in the way shown in the previous examples.
On line 21 the two fields are of the struct are joined together again with the \textit{CMergeCap} instruction.
This is not necessary, but we include it to show the working of the instruction.
The end of the program is the same as in the previous examples.
\begin{figure}[h]
\begin{lstlisting}[style=custASM, numbers = left ,xleftmargin=1.5em]
CMove          c2,  c20
li             x1,  0
sw.cap         x1,  0(c2)
li             x9,  4
CIncOffset     c2,  c2, x9
sw.cap         x1,  0(c2)   #let mut s = S{a:0,b:0};
CCreateToken   c31, c0      #{

CBorrowMut     c4, c2, c31
CSplitCap      c3, c2, x9   #a = &mut s.a;
                            #b = &mut s.b;

lw.cap         x1, 0(c3)
addi           x1, x1, 1
sw.cap         x1, 0(c3)    #*a += 1;

lw.cap         x1, 0(c2)
addi           x1, x1, 2
sw.cap         x1, 0(c2)    #*b += 2;

CMergeCap      c3,  c3, c2
CKillToken     c31, c31     #}
CRetrieveIndex c4,  c4, c31
\end{lstlisting}
\caption{An assembly program illustrating borrowing struct fields.}
\label{fig:asmstructexample}
\end{figure}

\subsection{Using Lifetime Fractions}


\subsection{Nested References}

%%% Local Variables:
%%% mode: latex
%%% TeX-master: "thesis"
%%% End:
