\documentclass[master=cws,masteroption=vs]{kulemt}
\usepackage{listings, listings-rust}
\usepackage{bytefield}
\usepackage{color}
\usepackage[usenames,svgnames,dvipsnames,acmlarge]{xcolor}
\usepackage{pdfpages}
\usepackage{amsmath,amssymb,amsfonts}
\usepackage{etoolbox}

\usepackage{xspace}
\newcommand{\perm}[1]{\textsc{\MakeLowercase{#1}}\xspace}
% intervals
\usepackage{interval}
\intervalconfig{ soft open fences } % use '(' and ')' for excluded bounds instead of '[' and ']'

\usepackage{mathpartir}
\usepackage{stmaryrd}



\usepackage{todonotes}
\newcommand\tomi[1]{\todo[author=Thomas, size=\small, color=red!20, inline]{#1}}
\newcommand\domi[1]{\todo[author=Dominique, size=\small, color=red!20, inline]{#1}}
\newcommand\tomit[1]{\todo[author=Thijs, size=\small, color=red!20, inline]{#1}}

\definecolor{mGreen}{rgb}{0,0.5,0} %very readable shade of green, used by Marco Patrignani in some of his paperss
\lstdefinestyle{custASM}{
  basicstyle=\footnotesize\ttfamily,
  escapeinside={/*@}{@*/},
  mathescape=true,
  commentstyle=\color{mGreen},
  tabsize=4,
  morecomment=[l]{\#},
  numberstyle=\tiny\color{gray},
  numbersep=6pt,
}

\setup{% Remove the "%" on the next line when using UTF-8 character encoding
  inputenc=utf8,
  title={Borrowed Capabilities: Ownership and Borrowing on a Capability Architecture},
  author={Thijs Vercammen},
  promotor={Prof.\,dr.\ D. Devriese\and Prof.\,dr.\,ir.\ F. Piessens},
  assessor={Ir.\ T. Van Strydonck\and Dr.\ K. Wuyts},
  assistant={Ir.\ T. Van Strydonck}}
% Remove the "%" on the next line for generating the cover page
%\setup{coverpageonly}
% Remove the "%" before the next "\setup" to generate only the first pages
% (e.g., if you are a Word user).
%\setup{frontpagesonly}

% Choose the main text font (e.g., Latin Modern)
\setup{font=lm}

% If you want to include other LaTeX packages, do it here. 

% Finally the hyperref package is used for pdf files.
% This can be commented out for printed versions.
\usepackage[pdfusetitle,colorlinks,plainpages=false]{hyperref}

\begin{document}
\selectlanguage{english}
\begin{preface}
I would like to thank everyone who supported me throughout the last year.
My supervisor Thomas for always giving me good advice and suggestions as well as making sure I actually kept working on my thesis.
My promotor Dominique for regularly checking in on my progress and helping me see the bigger picture.
My parents for taking care of me.
The rest of my family for the moral support.
My friends for keeping me company by playing online games together.

As this thesis also marks the end of my studies at KU Leuven, I would also like to thank all the people that I encountered throughout my time in Leuven.
My fellow students for being there with me.
All of the professors and other educators for their interesting (and not so interesting) courses.
All of the janitors and gardeners for making Campus Arenberg a nice place.
All of the other administrative KU Leuven staff for organizing everything.
The staff of Alma 3 for feeding me.
Mark Peeters for the interesting conversations at Alma 3.

And of course finally I would like to thank you for reading this thesis.
\end{preface}

\tableofcontents*

\begin{abstract}
Computer programs have been plagued by exploits and bugs related to manual memory management for a long time.
One way to prevent such issues is the use of programming languages with strong type systems such as Rust, which has risen in popularity in the past few years.
These languages use their type system restrictions to statically prevent runtime memory issues at compile time.
However when a program written in such a language is compiled, it usually gets compiled to an assembly language that doesn't have any of these safety guarantees.
This may open up the program to exploits if it is linked against a library that was written in a language that doesn't have the same safety guarantees.

Capability machine ISAs offer a security primitive called capabilities: unforgeable tokens that represent authority over memory.
Capabilities are very effective at preventing spatial memory issues at runtime, but they don't inherently provide countermeasures against the temporal memory issues that languages with strong type systems prevent.

In this thesis, we explore borrowed capabilities: a novel type of capability that is designed to mirror the Rust semantics at the ISA level.
The idea is to bind capabilities to a specific lifetime and allow those capabilities to be used only when a separate lifetime token with that lifetime is present in a specific register.
This allows software to create scopes in the form of lifetime tokens, bind capabilities to those scopes and end those scopes by unsetting a bit on the lifetime tokens.
In this context, borrowed capabilities could be seen as a way of providing a new general method for revocation on capability machines.
This thesis proposes a design for borrowed capabilities, an implementation in the ISA specification language Sail and an extension to the LLVM assembler.
The design is evaluated by constructing example assembly programs that mirror the behavior of certain simple Rust programs and by reasoning about the cost of implementing borrowed capabilities.
\end{abstract}

\selectlanguage{dutch} 
\begin{abstract}
Computerprogramma's worden al lange tijd geplaagd door kwetsbaarheden en bugs die relateren tot handmatig geheugenbeheer.
Een manier om zulke problemen te voorkomen is het gebruik van programmeertalen met sterke typesystemen zoals Rust, dat in de afgelopen paar jaar populair is geworden.
Deze talen gebruiken de beperkingen opgelegd door hun typesystemen om op een statische manier runtime geheugenproblemen op het moment van compileren te voorkomen.
Wanneer een programma geschreven in zo'n taal echter wordt gecompileerd, wordt het gewoonlijk gecompileerd naar een assemblytaal die geen van deze veiligheidsgaranties heeft.
Dit kan het programma open laten voor kwetsbaarheden als het gelinkt is tegen een library die geschreven is in een taal zonder dezelfde veiligheidsgaranties.
Capability machine ISAs bieden een veiligheidsprimitief genaamd capabilities aan: onvervalsbare tokens die authoriteit over geheugen voorstellen.
Capabilities zijn zeer effectief in het voorkomen van ruimtelijke geheugenproblemen, maar ze bieden geen inherente tegenmaatregelen aan tegen tijdelijke geheugenproblemen die talen met sterke typesystemen voorkomen.

In deze thesis onderzoeken we borrowed capabilities, een nieuw type van capability dat ontworpen is om Rust semantiek op het niveau van de ISA voor te stellen.
Het is idee is om capabilities aan een specifieke lifetime te binden en om die capabilities enkel gebruikt te laten worden wanneer een apart lifetime token met hun lifetime aanwezig is in een specifiek register.
Dit laat software toe om scopes te definiëren in de form van lifetime tokens, capabilities aan deze scopes te binden en de scopes te beëindigen door een bit op de lifetime tokens af te zetten.
In deze context kunnen borrowed capabilities gezien worden als een nieuwe algemene manier om revocatie op capability machines te voorzien.
Deze thesis stelt een design voor borrowed capabilities voor, een implementatie in de ISA specificatietaal Sail en een uitbreiding aan de LLVM assembler.
Het design wordt geëvalueerd door voorbeeld assembly programma's te construeren die het gedrag van bepaalde simpele Rust programma's nabootsen en door te redeneren over de kost van het implementeren van borrowed capabilities.
\end{abstract}
\selectlanguage{english}

% A list of figures and tables is optional
%\listoffigures
%\listoftables
% If you only have a few figures and tables you can use the following instead
\listoffiguresandtables
% The list of symbols is also optional.
% This list must be created manually, e.g., as follows:
\chapter{List of Abbreviations}
\section*{Abbreviations}
\begin{flushleft}
  \renewcommand{\arraystretch}{1.1}
  \begin{tabularx}{\textwidth}{@{}p{12mm}X@{}}
    DDC   & Default Data Capability \\
    ISA   & Instruction Set Architecture \\
    OBS   & Ownership and Borrow System \\
    PCC   & Program Counter Capability \\
    TCB   & Trusted Computing Base \\
  \end{tabularx}
\end{flushleft}

% Now comes the main text
\mainmatter

\chapter{Introduction}
\label{cha:intro}
For many years computer programs have been plagued by vulnerabilities that can be exploited in order to gain control over the execution of a program.
Several classes of these vulnerabilities arise from bugs in the source code and are related to low level control over memory.
Spatial memory issues are vulnerabilities that arise from accessing memory that is outside of the bounds that are supposed to be accessed.
Temporal memory issues on the other hand arise from accessing memory after the reference to that memory is supposed to have been invalidated.
Some famous examples of spatial and temporal memory issues are buffer overflows and dangling pointers whereby an attacker can write to memory that is not supposed to be written to \cite{van_der_veen_memory_2012}.

Over the past decades, several solutions to these issues have been presented and implemented.
One of the most popular solution is the usage of memory safe programming languages that deny the programmer low level access to the memory \cite{10.1145/780731.780743}.
Instead, the memory is managed by a language runtime and deallocated by a garbage collector.
This solution, however, has a significant performance impact \cite{10.1145/1094811.1094836} and is thus not applicable to performance sensitive programs such as kernels, browsers and low level system libraries.
These types of programs remain vulnerable and are often a vector for malware to gain control over a system.
A class of devices that are especially vulnerable to these issues are embedded devices, since they have constrained resources and programmers require low level memory management features to gain reasonable performance.
With the advent of the Internet of Things (IoT), these embedded devices are expected to increase in number dramatically \cite{8688434}.

Over the past few years programming languages with strong type systems like Rust have gained attention as a way to mitigate these issues and improve security \cite{10.1145/2692956.2663188}.
Rust offers a number of security features, but in this thesis we will focus on Rust's ownership and borrowing system.
This system limits what a programmer can do with references to an object in memory, ensuring that memory is always accessed in a safe manner, and statically guaranteeing thread safety.
One of the most important principles of this system is the principle of ``aliasing XOR mutability'' which means that references to a resource in memory are either allowed to mutate the resource or are read-only and can be copied, but not both.
This prevents concurrent write accesses and concurrent read and write access to a resource in memory, which ensures the absence of things like data races.

However, when a Rust program is compiled, it is usually converted into an unsafe assembly language that does not have any of these safety guarantees.
This might, for example, open the program up to attacks on the assembly level if it is linked with libraries that are not written in a programming language with the same safety guarantees.
In this thesis we explore an extension to the capability machine architecture CHERI in order to provide support on the assembly level for the safety guarantees that the Rust programming language offers.
This could then be used as the target for a \textit{secure compiler}, a compiler that ensures that any attack against the target language is possible if and only if it is possible against the source language \cite{8049734}.

Hardware capabilities are unforgeable pointers that represent authority over a region of memory.
They are comparable to software fat pointers in that they enforce bounds checking and hold permissions that specify how a region of memory can be used.
This model is very effective at combating spatial memory issues, but is less suited to enforce the temporal memory safety guarantees offered by languages with strong type systems.
Nevertheless, CHERI does offer some hardware primitives that software can build upon to strengthen their defense against temporal memory issues.
In this thesis we will add a new type of capability called \textit{borrowed capability}.
This new type of capability is designed to mimic Rust's ownership and borrowing system, but it could be seen in a wider context as an addition to the types of revocation that are already offered by CHERI.
In this context, borrowed capabilities would allow software to define scopes that capabilities can be bound to and revoking a scope would revoke all the capabilities bound to it.
This type of revocation would fit well into CHERI as it would be more specific then some existing types revocation while being more flexible than others.

In summary, the goal of this thesis is to design an extension to the CHERI architecture that offers the same memory and thread safety guarantees as the Rust programming language.
To test our design, we will implement it in the Sail specification language, extend the LLVM project to create an assembler for our extension and run test programs on the emulator generated by Sail.
We start off by giving background information about Rust and CHERI as well as information about RISC-V and the Sail language which will be essential for the implementation of our design in chapter \ref{cha:litrev}.
Next, in chapter \ref{chap:design} we provide an overview of our design, followed by the details of its implementation in Sail in chapter \ref{chap:sailimpl}.
In chapter \ref{chap:assemblerimpl} we will give the details of the LLVM extension.
We will then use the emulator that is generated by our Sail implementation and the assembler to evaluate our design by constructing and executing a number of assembly example programs and reason about how well they mimic the Rust ownership and borrowing system in chapter \ref{chap:evaluation}.
Finally, we finish with a discussion about various aspects of borrowed capabilities in chapter \ref{chap:discussion} and the conclusion in chapter \ref{cha:conclusion}.

A short paper that contains a summary of the work in this thesis was submitted to SILM 2021 for the workshop on the security of software / hardware interfaces and is attached in appendix \ref{app:paper}.

%%% Local Variables: 
%%% mode: latex
%%% TeX-master: "thesis"
%%% End: 

\chapter{Literature Review}
\label{cha:litrev}
The topic of this thesis lies at the intersection of multiple different fields including programming languages, capability machines, formal software verification and instruction set architecture (ISA) design. Understanding this thesis requires background knowledge of each of these individual subjects.

\section{The Rust programming language}
\label{sec:rustbackground}
Rust is a relatively new programming language that aims to provide the high performance associated with other low-level programming languages such as C and C++ while simultaneously preventing a number of classes of bugs related to memory and thread safety that typically arise due to the low-level resource management features necessary to achieve such high performance.
Rust accomplishes this at compile time by statically checking if the source code satisfies the rules of Rust's strong ownership-based type system.
Because these rules are conservative and might reject correct programs, Rust provides an \textit{unsafe} mode in which some of these rules do not need to be satisfied.
Since this thesis is about the safety Rust provides, we only consider \textit{safe} Rust in this section.

\subsection{Ownership}
Central to Rust's resource management model is the concept of ownership.
When a value gets created, ownership of that value will be assigned to a variable.
This value stays valid until its owner goes out of scope after which it cannot be used anymore.
Ownership of a value can be transferred to another variable, when this happens the value cannot be accessed anymore through the original variable.
This is illustrated in listing \ref{code:move} which fails to compile because \textit{a} gets accessed after its value has been moved to \textit{b}.

\begin{figure}[h]
\begin{lstlisting}[language=Rust,frame=single,caption=Moving a variable,label=code:move]
fn main() {
  let a = String::from("This is a heap allocated string");
  let b = a;
  println!("{}", a);
}
\end{lstlisting}
\end{figure}

\subsection{Borrowing}
\label{subsec:borrowing}
Because moving ownership back and forth between functions is impractical and prevents the programmer from having multiple read-only copies of a resource, Rust has a system to create additional references to a resource, called the borrowing system.
As mentioned, Rust follows the principle of ``aliasing XOR mutation'' to prevent a number of bugs related to memory safety and data races.
The borrowing system follows a set of rules for references that enforces this principle.
A value can be borrowed by variables other than its owner, giving those variables the temporary right to use the value.
There are two types of borrows: a mutable borrow and an immutable (or shared) borrow. 

A mutable borrow allows the borrowing variable to both read and mutate the contents of the value.
To prevent multiple variables mutating the same value at the same time and to prevent variables from reading a value that is being altered, the owner temporarily loses its right to access or lend the value while a mutable borrow is in scope.
As a result, only the borrowing variable is allowed to access the value while a mutable borrow exists.
Of course, in order to make a mutable borrow, the owner variable must also be mutable by being by being declared \texttt{mut}.

A shared borrow only allows a borrowing variable to read the contents of the value.
While a shared borrow is in scope, the owner cannot mutate or mutably lend the value anymore.
It is however allowed to read the value and immutably lend the value again.
This means that an unlimited amount of shared borrows can read a value at the same time, but it cannot be mutated while any of those shared borrows exist.

Listing \ref{code:borrowexample} shows a simple program that creates a mutable variable \textit{x}, borrows it mutably and increments the value through that borrow 
Then, after the mutable borrow has ended, the program borrows \textit{x} twice, but this time immutably and uses those borrows to print the value. Note that during the immutable borrow, the owner variable \textit{x} stays available to read from.

\begin{figure}[h]
\begin{lstlisting}[language=Rust,frame=single,caption=Borrowing an integer,label=code:borrowexample]
fn main() {
  let mut x = 5;
  {
    let m = &mut x;
    *m += 1;
  }
  {
    let s1 = &x;
    let s2 = &x;
    println!("{}, {}, {}",x, s1, s2);
  }
}
\end{lstlisting}
\end{figure}

\subsection{Lifetimes}
Section \ref{subsec:borrowing} explained the rules regarding borrowing, but did not explain how a borrow actually ends.
For this, Rust has a concept called lifetimes.
Every borrow has a lifetime that roughly corresponds to its variable scope which determines when the borrow is valid.
In order to lighten the load on the programmer, the Rust compiler usually implicitly infers the lifetimes of borrows.
However, in some situations it is necessary to explicitly annotate the Rust code with lifetimes.
Listing \ref{code:lifetime_semantics} shows an example from the Rust book \cite{rustbook} with such a situation.
In this example, the function \textit{longest} takes two string references as arguments and returns a reference to one of those.
The function does not know which arguments it will be called with and thus cannot implicitly infer any information about their lifetimes.
This also means it cannot give any guarantees about the lifetime of its return value.
This is why it is necessary to link the arguments to the return value using lifetime parameter annotations.
These annotations allow the function to tell its caller that the returned reference has the same lifetime as the arguments that were passed in.

\begin{figure}[h]
\begin{lstlisting}[language=Rust,frame=single,caption=Lifetime example,label=code:lifetime_semantics]
fn main() {
  let string1 = String::from("abcd");
  let string2 = "xyz";

  let result = longest(string1.as_str(), string2);
  println!("The longest string is {}", result);
}

fn longest<'a>(x: &'a str, y: &'a str) -> &'a str {
  if x.len() > y.len() {
    x
  } else {
    y
  }
}
\end{lstlisting}
\end{figure}

\subsection{Semantic Lifetimes}
\label{sec:semantic_lifetimes}
In the previous section we mentioned that lifetimes roughly correspond to scopes.
In the early versions of Rust, lifetimes were indeed bound to scopes, but in more recent versions of Rust the compiler is smart enough to detect the last usage of a variable and ends the lifetime after that last usage.
This behavior is called \textit{semantic lifetimes} because it takes the semantic meaning of the program into account when determining the lifetime of a variable.
This behavior means that code like in listing \ref{code:semantic_lifetime} compiles correctly.
Even though the mutable borrow $m$ is still in scope when $x$ gets incremented, the compiler detects that the increment to $m$ is the last usage of $m$ and thus ends its lifetime after that line which allows $x$ to be used again.
In this thesis we will usually place borrows in their own scope when giving examples for clarity reasons.

\begin{figure}[h]
\begin{lstlisting}[language=Rust,frame=single,caption=Semantic lifetimes,label=code:semantic_lifetime]
fn main() {
  let mut x = 5;
  let m = &mut x;
  *m += 1;
  x += 1;
}
\end{lstlisting}
\end{figure}

\subsection{Borrowing Specifics}
Borrowing has a number of interesting implications depending on the situation and the data types that get borrowed.
Because these interactions are important when we model the Rust semantics on the hardware level, we introduce them in more depth here.

\subsubsection{Reborrowing}
\label{sec:backgroundreborrow}
In section \ref{subsec:borrowing} we stated that a value can not be immutably borrowed while it is mutably borrowed.
This is not entirely accurate as listing \ref{code:reborrow_semantics} shows.
In the example we create a mutable value and borrow it mutably.
We then create a shared borrow from the mutable borrow.
This phenomenon where a borrow gets borrowed again is called a reborrow.
Reborrowing a borrow works similarly to borrowing an owner in the sense that the original borrow --- which we will also call the \textit{parent} --- loses (some of) its permissions while the reborrowed reference --- which we will also call the \textit{child} --- is active.
This can be seen in the difference between listing \ref{code:reborrow_semantics} and listing \ref{code:reborrow_semantics_wrong}.
In listing \ref{code:reborrow_semantics}, the mutable borrow is not used until the lifetime of the shared borrow has already ended and the code compiles correctly.
However, if we move the increment to the value up as in Listing \ref{code:reborrow_semantics_wrong}, the Rust compiler returns an error because the shared borrow still gets used after the increment.
This reborrow feature adds a lot of expressivity to Rust's fairly strict semantics and will have a major impact on our design choices for borrowed capabilities.

\noindent
\begin{tabular}{p{6.65cm} p{6.65cm}}
    \begin{lstlisting}[language=Rust,frame=single,caption=Reborrow example,label=code:reborrow_semantics]
fn main() {
  let mut x = 5;
  {
    let m = &mut x;
    {
      let s = &(*m);
      println!("{}", s);
    }
    *m += 1;
  }
}
    \end{lstlisting}

    &

    \begin{lstlisting}[language=Rust,frame=single,caption=Wrong Example,label=code:reborrow_semantics_wrong]
fn main() {
  let mut x = 5;
  {
    let m = &mut x;
    {
      let s = &(*m);
      *m += 1;
      println!("{}", s);
    }
  }
}
    \end{lstlisting}
\end{tabular}

\subsubsection{Compound Data Types}
While safe Rust does have a large amount of limitations around slices of data types such as arrays, the borrowing of individual fields of structs is well supported.
This means that it is possible to have multiple mutable references or both mutable and shared references to the same struct at the same as long as they point to different members of the struct.
In a more general sense, such named regions in memory are often referred to as \textit{places} \cite{2019arXiv190300982W}.
Rust allows places within compound data structures to be borrowed as long as they do not overlap.
Listing \ref{code:struct_semantics} shows an example where both fields of a struct get mutably borrowed and independently altered.
This code works and is considered safe Rust because the borrows point to disjoint places.

\begin{figure}[h]
\begin{lstlisting}[language=Rust,frame=single,caption=Borrowing struct fields,label=code:struct_semantics]
struct S {
  a: i32,
  b: i32
}

fn main() {
  let mut s = S{a: 0, b: 0};
  {
    let a = &mut s.a;
    let b = &mut s.b;
    *a += 1;
    *b += 2;
    println!("{}, {}", a, b);
  }
}
\end{lstlisting}
\end{figure}

\subsubsection{Nested References}
\label{sec:rust_nested}
Another problem occurs when for example compound data structures hold references to other resources.
Rust has some complex rules surrounding the loading of references through a borrow that points to a data structure that contains so-called ``nested references''.
In this section we will explain some of these peculiarities through some simple examples.

The first rule pertains to the loading of references through a borrow in general.
Listing \ref{code:nested_move} shows an instantiation \textit{s} of a struct that holds a reference to an integer.
The definition of the struct \textit{S} has a lifetime parameter which indicates that an instance of the struct cannot outlive the reference \textit{n} within it.
Then, \textit{s} gets borrowed by the variable \textit{b} and the variable \textit{n} tries to load the reference to the integer.
This, however, fails because this code tries to move the reference out of the struct into \textit{n} which would leave the struct with an empty \textit{n} field.
The simple solution to this issue is to borrow the reference to the integer instead, as shown in listing \ref{code:nested_borrow}.
While this may seem like a trivial difference, it is important to note because it means that ownership of a reference cannot be moved out of a struct via a borrow and it means that the variables that have loaded such a reference are bound to a sub-lifetime of the lifetime of the nested reference themselves.
\begin{tabular}{p{6.65cm} p{6.65cm}}
    \begin{lstlisting}[language=Rust,frame=single,caption=Move reference,label=code:nested_move]
struct S<'a> {
  n: &'a mut i32
}

fn main() {
  let mut x= 0;
  let mut s = S{n: &mut x};
  let b = &mut s;
  let n = b.n;
}
    \end{lstlisting}

    &

    \begin{lstlisting}[language=Rust,frame=single,caption=Borrow reference,label=code:nested_borrow]
struct S<'a> {
  n: &'a mut i32
}

fn main() {
  let mut x= 0;
  let mut s = S{n: &mut x};
  let b = &mut s;
  let n = &b.n;
}
    \end{lstlisting}
\end{tabular}

Our next example in listing \ref{code:nested_immutmut} shows that \textit{s} is immutably borrowed by \textit{b} and \textit{n} then tries to get a mutable reference through \textit{b}.
This operation, however, fails to compile because Rust does not allow the loading of mutable references through shared references.
The reason for this is that there could be multiple shared references that could be used to load the mutable reference and this behavior would violate the ``aliasing XOR mutation'' principle.

\begin{figure}[h]
\begin{lstlisting}[language=Rust,frame=single,caption=Mutable reference through shared borrow,label=code:nested_immutmut]
struct S<'a> {
  n: &'a mut i32
}

fn main() {
  let mut x= 0;
  let mut s = S{n: &mut x};
    
  let b = &s;
  let n = &mut b.n;
}
\end{lstlisting}
\end{figure}

We will not support all of these rules in our design for borrowed capabilities.

\subsection{Threading}
Thread safety is one of the major features of Rust as exemplified by the Rust slogan ``fearless concurrency''.
In our implementation of borrowed capabilities we will pay attention to safe threading.
Listing \ref{code:thread} shows an example of how threading works in Rust.
We start off by wrapping an integer in the \textit{Arc} construct.
This is necessary because threading in the Rust standard library has no concept of scopes which the Rust borrow checker assumes that created threads outlive their parent thread and thus the values that were owned by the parent even if the programmer correctly waits for the child threads to terminate.
The \textit{Arc} construct works around this by ensuring that a value is not deallocated until each owner of an \textit{Arc} reference is out of scope.
Using a third party threading library with thread scopes would remove the need for the \textit{Arc} construct.
After creating two \textit{Arc} references to the same integer value, we move each one of them into a new thread and then wait for those threads to finish.
Rust ensures that principle of ``aliasing XOR mutation'' is upheld by prevent the mutation of the value inside the \textit{Arc} construct.

\begin{figure}[h]
\begin{lstlisting}[language=Rust,frame=single,caption=Threading in Rust,label=code:thread]
use std::thread;
use std::sync::Arc;

fn main() {
  let x = Arc::new(5);
  let c = x.clone();

  let handle1 = thread::spawn(move || {
    println!("{}", x);
  });

  let handle2 = thread::spawn(move || {
    println!("{}", c);
  });
  
  handle1.join().unwrap();
  handle2.join().unwrap();
}
\end{lstlisting}
\end{figure}

\subsection{Rust Borrow Invariants}
\label{sec:obsinvariants}
In this section we will give a more in-depth explanation of what the Rust safety guarantees are in the form of a set of invariants.
Kan et al.\ \cite{kan2020executable} identify a set of five invariants that must hold when borrowing in Rust.
These invariants that they refer to as the ownership and borrowing system (OBS) invariants are listed in figure \ref{fig:obsinvariants}.
The OBS invariants ensure that each reference points to a valid block of memory and that each valid block can either be accessed through multiple immutable references or one mutable reference.

\begin{figure}[h]
\centering
\begin{enumerate}
    \item \textit{Unique owner invariant}: Each block has a unique owner.
    \item \textit{Lifetime inclusion invariant}: If $x$ borrows or reborrows $y$ then the lifetime of $x$ should always be within the lifetime of $y$ in order to avoid dangling pointers.
    \item \textit{Lifetime disjoint invariant}: There are \textit{no} two references to the same referent such that their lifetimes intersect and one of them is a mutable reference.
    \item \textit{Writing permission invariant}: If $x$ borrows or reborrows $y$ immutably then the writing permission of $y$ should be disabled until the end of $x$'s lifetime.
    \item \textit{Reading and writing permissions invariant}: If $x$ borrows or reborrows $y$ mutably then both the reading and writing permissions of $y$ should be disabled until the end of $x$'s lifetime.
\end{enumerate}
\caption{The five OBS invariants.\cite{kan2020executable}}
\label{fig:obsinvariants}
\end{figure}

\subsection{Rust Static Guarantees}
Because the compiler automatically deallocates values after their owner goes out of scope, the programmer cannot make typical deallocation related mistakes such as \textit{double free} or \textit{invalid free}.
An owner can only go out of scope when no more borrows exist, meaning that a \textit{use after free} bug is impossible.
The borrow system eliminates the necessity of using raw pointers which prevents problems like buggy pointer arithmetic.
Rust also prevents concurrent read and write access to a value, making data races impossible.
All of these static safety guarantees also result in Rust's ability to prevent thread related bugs and enforce thread safety.
Jung et al.\ \cite{10.1145/3158154} have built a formal model that resembles Rust and serves as a basis for proving Rust's safety guarantees.

%%% Local Variables:
%%% mode: latex
%%% TeX-master: "../thesis"
%%% End:

%\section{CHERI}
Capability Hardware Enhanced RISC Instructions (CHERI) is an architecture neutral model of hardware supported capability machines, largely developed by researchers at Cambridge University.\cite{UCAM-CL-TR-951} This chapter starts off with a general introduction to capabilities which is followed by a deeper look into the specifics of the CHERI protection model and its implementation in the RISC-V ISA.

\subsection{Capabilities}
\label{sec:capintro}
Capabilities are unforgeable tokens that give a process access to some resource. They are created by a privileged entity such as the hardware or the kernel. This privileged entity also checks and enforces the correct use of capabilities. In the context of memory they are a method of addressing memory that is different from the traditional approach of using raw integer pointers. Owning a memory capability gives a process the rights to read from, write to or execute the contents of a location in memory, depending on the access permissions of the capability. Memory capabilities can usually give access to a range of memory locations which allows them to properly express concepts like arrays and structs. Figure \ref{fig:capability} depicts a capability layout from an early iteration of the CHERI design and allows for a simple explanation of how capabilities work.

\begin{figure}[h]
\centering
\definecolor{lightgray}{gray}{0.8}
\begin{bytefield}[endianness=big, bitwidth=.55em]{64}
    \bitheader{0,63} \\
    \bitbox{8}{\color{lightgray}\rule{\width}{\height}} & \bitbox{24}{\textbf{otype} (24 bits)} & \bitbox{31}{\textbf{permissions} (31 bits)} & \bitbox{1}{\textbf{s}} \\
    \bitbox{64}{\textbf{offset} (64 bits)} \\
    \bitbox{64}{\textbf{base} (64 bits)} \\
    \bitbox{64}{\textbf{length} (64 bits)}
\end{bytefield}
\caption{An old version of the CHERI capability layout.\cite{Watson2015CHERIAH}}
\label{fig:capability}
\end{figure}

A process that holds a capability like in figure \ref{fig:capability} is able to access the memory locations between \textit{base} and \textit{base+length} by modifying the \textit{offset} so that \textit{base+offset} points to the desired memory address. The process can freely modify the \textit{offset} field, but it is only allowed to alter the \textit{base} or \textit{length} fields in such a way that they point to a subset of the memory that the current capability points to. If the process tries to dereference the capability when the \textit{offset} field is set in such a way that it does not point to a memory location within the permissible range, some sort of error will be raised, such as a hardware exception in the case of CHERI. This prevents a whole range of possible security issues that are present in systems using raw pointers such as faulty pointer arithmetic and buffer overflows.

The \textit{permissions} field contains a bitmap of various permissions such as read, write and execute. Modifying the \textit{permissions} field works similar to modifying the \textit{base} and \textit{length} fields in the sense that the process can only modify them in such a way that results in less permissions than the current capability. This form of monotonicity where a process can only reduce the rights on the capability it holds is central to the CHERI design. It gives a process a method to efficiently restrict the rights on the capabilities it delegates to components it may not trust while still retaining those rights for itself (by holding a copy of the original capability). The \textit{otype} and \textit{sealed} (\textit{s}) fields will be explained in section \ref{sec:sealed}.

\subsection{CHERI Capability Layout}
\label{sec:cheri_cap_layout}
Figure \ref{fig:cheri_capability} shows the layout of 128-bit capabilities in the latest iteration of the CHERI specification. This capability format is intended for use in 64-bit architectures. While a 64-bit capability format intended for use in 32-bit architectures exists, this format is too limited to support the work in this thesis, so we will not consider it here.

\begin{figure}[h]
\centering
\definecolor{lightgray}{gray}{0.8}
\begin{bytefield}[endianness=big, bitwidth=.55em]{64}
    \bitheader{0,63} \\
    \bitbox{14}{\textit{p}'16} & \bitbox{3}{\color{lightgray}\rule{\width}{\height}} & \bitbox{16}{otype'18} & \bitbox{3}{\textit{$I_E$}} & \bitbox{8}{\textit{T}[11:3]} & \bitbox{5}{\textit{$T_E$}'3} & \bitbox{10}{\textit{B}[13:3]} & \bitbox{5}{\textit{$B_E$}'3} \\
    \bitbox{64}{\textit{a}'64}
\end{bytefield}
\caption{The current iteration of the CHERI capability layout.\cite{UCAM-CL-TR-951}}
\label{fig:cheri_capability}
\end{figure}

These capabilities consist of two 64-bit words with a number of fields:

\begin{itemize}
    \item p: The permissions field is a bitvector containing various permissions, explained in more detail in section \ref{sec:capperms}. It consists of two subfields:
        \begin{itemize}
            \item hperms: The hardware permissions field is 12 bits wide and contains the permissions defined by CHERI such as read, write, execute and various others.
            \item uperms: The user permissions field is 4 bits wide and is reserved for application specific uses.
        \end{itemize}
    \item reserved: There are 3 unused bits reserved for possible future or implementation specific use.
    \item otype: The object type field is used by sealed capabilities as explained in section \ref{sec:sealed}.
    \item bounds: The next 27 bits are used to define the bounds of the capability. This is a compressed representation of the \textit{base} and \textit{length} fields described in section \ref{sec:capintro}. The bounds bits are explained in more detail in section \ref{sec:bounds}.
        \begin{itemize}
            \item $I_E$: the internal exponent.
            \item E: the exponent.
            \item T: the top.
            \item B: the base.
        \end{itemize}
    \item a: The address field holds the referenced memory location. This field is similar to a raw integer pointer.
\end{itemize}

\subsection{Capability Tags}
In CHERI, each register and each capability-aligned memory location is associated with a 1-bit tag that is kept out of band. This means that the tag cannot be directly accessed by the software. A tag indicates the presence of a valid capability and is read from and written to in certain hardware operations. The tag allows the hardware to guarantee that the software does not alter its capabilities in a way that is not allowed. In general, if a program wants to alter a capability in any way, it needs to do this through CHERI capability manipulation instructions. If the program writes to a memory location or register where a capability resides using traditional write instructions, the hardware will clear the tag which invalidates the associated capability. It is then no longer considered a valid capability and it cannot be used anymore by the majority of CHERI instructions. It is the responsibility of the compiler to ensure that no necessary capabilities are accidentally invalidated. 

\subsection{PCC \& DDC}
Two of the most important special registers added by CHERI are the program counter capability (PCC) and the default data capability (DDC). The PCC extends the traditional program counter as a capability which allows software to limit the fetching of instructions to memory locations for which the application holds a capability with execute permissions.

The DDC register holds a capability through which traditional, non-capability memory accesses are indirected. This is a hybridization feature that allows for running non-CHERI compiled binaries on a CHERI CPU with some of the CHERI safety guarantees. Every legacy load or store instruction is checked against the bounds and permissions of the capability in the DDC register. If the memory access is allowed by the capability in the DDC register, the instruction completes successfully. If however the memory access is not allowed, the instruction that is attempting to access memory will trap.

\subsection{Capability Permissions}
\label{sec:capperms}
The permissions field is a simple bitvector where each bit corresponds to a certain permission. An application can manipulate permissions on their capabilities through the \textit{CAndPerm} instruction. The \textit{CAndPerm} instruction takes a capability and a bitmask as arguments and outputs a capability with that bitmask applied to the permissions field. This operation ensures that an application cannot give itself more permissions since a bitmask can not be used to set a bit that was not originally set. This means that the capability rights monotonicity is preserved.

The hardware permissions (\textit{hperms}) subfield is a 12-bit field that contains the permissions that are interpreted by the hardware. Given below is the interpretation of each bit in the vector, starting from the least significant bit. The \textit{Global} and \textit{Permit\_Store\_Local\_Capability} permissions are further explained in section \ref{sec:global}. The \textit{Permit\_Seal}, \textit{Permit\_CInvoke} and \textit{Permit\_Unseal} permissions are further explained in section \ref{sec:sealed}.
\begin{itemize}
    \item Global: Allow this capability to be stored by all capabilities that have the \textit{Permit\_Store\_Capability} permission.
    \item Permit\_Execute: Allow this capability to be used to fetch instructions.
    \item Permit\_Load: Allow this capability to be used to load data from memory.
    \item Permit\_Store: Allow this capability to be used to store data in memory.
    \item Permit\_Load\_Capability: Allow this capability to be used to load capabilities from memory. The \textit{Permit\_Load} bit is also required.
    \item Permit\_Store\_Capability: Allow this capability to be used to store capabilities in memory. The \textit{Permit\_Store} bit is also required.
    \item Permit\_Store\_Local\_Capability: Allow this capability to be used to store capabilities that do not have the \textit{Global} bit set. The \textit{Permit\_Store} and \textit{Permit\_Store\_Capability} bits are also required.
    \item Permit\_Seal: Allow this capability to be used to seal other capabilities.
    \item Permit\_CInvoke: Allow this capability to be used as an argument to CInvoke.
    \item Permit\_Unseal: Allow this capability to be used to unseal other capabilities.
    \item Permit\_Set\_CID: Allow this capability to be used to set the architectural compartment ID.
    \item Access\_System\_Registers: Allow this capability to be used to access privileged registers. The interpretation of this bit is architecture-specific.
\end{itemize}
The 4-bit user permissions field (\textit{uperms}) is intended for application specific use and is not interpreted by the hardware. These bits follow the same principles as the \textit{hperms} bits, but an application can give its own semantic meaning to them.

\subsection{Global and Local Capabilities}
\label{sec:global}
A capability can either be \textit{global}, which means it can be stored in memory by any capability that has the \textit{Permit\_Store\_Capability} permission, or \textit{local}, which means it can only be stored by capabilities that have the \textit{Permit\_Store\_Local\_Capability} permission. These low level permissions can be used to construct higher level restrictions on the propagation of capabilities depending on which capabilities have the \textit{Global} and \textit{Permit\_Store\_Local\_Capability} bits set. This propagation restriction depends on different components of software having access to different parts of memory and having access to capabilities with differing \textit{Permit\_Store\_Local\_Capability} permissions.

For example, if a component has exclusive access to some piece of memory (i.e. no other component has a capability pointing to that part of memory) through a capability with \textit{Permit\_Store\_Local\_Capability} and it has no other capabilities with \textit{Permit\_Store\_Local\_Capability}, it means that \textit{local} capabilities cannot be passed on to other components. As another example, consider a component that has a capability pointing to a shared part of memory with \textit{Permit\_Store\_Local\_Capability}. This means that the component can store \textit{local} capabilities to that shared part of memory. Other components can then load the \textit{local} capability from the shared memory, but they cannot store it again, unless they have a capability with \textit{Permit\_Store\_Local\_Capability} themselves.

Some possible applications for this system described in the CHERI specification are restricting local stack capabilities to be stored only on the local stack or letting capabilities flow in only one way through a shared buffer.\cite{UCAM-CL-TR-951} %page 59

\subsection{Linear Capabilities}
Linear capabilities is a concept that has been proposed by multiple researchers and is currently being considered for adoption in CHERI. We briefly introduce it here because the work in this thesis makes extensive use of linear capabilities. Linear capabilities are capabilities that can not be copied in any way.
This is useful to implement mechanisms such as Rust's ownership system.
It also guarantees the holder of the linear capability that no other party can access the memory that the capability points to, making it safe for the holder to do whatever it wants with the memory.
When a caller delegates a linear capability to a callee and the callee then returns the linear capability, the caller can be sure that the callee does not have access to the memory anymore, acting as a form of \textit{revocation}.

Two proposed instructions that relate to the splitting and merging of linear capabilities with congruent sections of memory have been found to be difficult to implement in microarchitecture because they have to write to two different registers in the same instruction. For the work in this thesis we rely on these instructions. We acknowledge these difficulties as a weakness in our design, but do not attempt to mitigate this issue.

\subsection{Revocation}
\label{sec:backgroundcherirevocation}
Capabilities are particularly well suited to ensure spatial memory safety, but they are not inherently designed to solve temporal memory safety issues.
One important concept in preventing temporal memory safety issues is \textit{revocation}; repealing the access an untrusted stakeholder has to a certain resource.
Prior work has attempted to solve the revocation issue in several ways.
CHERIvoke \cite{xia_cherivoke_2019} and Cornucopia \cite{nathaniel_wesley_filardo_cornucopia_2020} present a method of revoking capabilities by modifying the system's memory allocator and periodically sweeping memory to remove capabilities pointing to memory that has been freed.
While this approach to temporal safety could be tailored to different scenarios than just memory allocation, it includes the memory allocator into the trusted computing base (TCB), and could be prohibitively expensive if the memory sweep is required often (e.g.\ after each call to an adversary).
Two other approaches each use \textit{local} and \textit{linear} capabilities to implement a type of revocation that does not rely on a memory sweep or a software TCB, and could perform better in the aforementioned scenarios.

Skorstengaard et al.\ introduce a new calling convention using local capabilities \cite{skorstengaard:esop18} to enforce revocation on the stack.
Georges et al. have proposed a new type of capability, called uninitialized capabilities \cite{georges_efficient_2021} to prevent the necessity of clearing the write-local memory in this stack setting.
Van Strydonck et al.\ used linear capabilities to implement revocation in a fully abstract compiler from separation-logic-verified C code to a capability machine \cite{van_strydonck_linear_2019}, and Skorstengaard et al.\ used them to enforce well-bracketed control flow and stack encapsulation \cite{skorstengaard:popl19}.
Because linear capabilities cannot be copied at all, they are a very restrictive way of revoking capabilities.
For example, they cannot be used to provide multiple parties with simultaneous read-only access, or in a multi-threaded setting.

Our design of borrowed capabilities will provide a new way of revocation with some advantages compared to local and linear capabilities which we will discuss in more depth in chapter \ref{chap:discussion}.

\subsection{Sealed Capabilities}
\label{sec:sealed}
In their most basic form, sealed capabilities are capabilities that cannot be modified or dereferenced. Normal unsealed capabilities are identified by an otype value of $2^{18} - 1$; for all other values of the 18-bit \textit{otype} field, the capability is considered sealed. CHERI currently reserves 16 otype values for specific interpretations, all other otype values can be used for code-data pairs, further explained below. The reserved otype values range from $2^{18} - 1$ to $2^{18} - 16$.The first of these values is used for unsealed capabilities, as mentioned above. The second one is used for sealed entry (sentry) capabilities, the other values in the range have currently not been assigned. Sentry capabilities are intended to be used as pointers to code that can be executed and jumped to by control flow instructions. The modification restriction of sealed capabilities fits this use case well because it prevents applications from creating unintended control flows.

The main use case for sealed capabilities and the \textit{otype} field is the concept of code-data pairs. Code-data pairs are designed to support compartmentalization of software within a single address space. As their name implies, a code-data pair is a pair of two capabilities, one of them pointing to the code of a compartment and the other pointing to the data used by this compartment. Code-data pairs are a form of object capabilities which are essentially closures: encapsulated objects that contain code to execute and an environment, represented by the code and the data capability respectively.

Code-data pairs are created by sealing both capabilities with the CSeal instruction. This instruction takes two input operands. The first one is the code or data capability to be sealed and needs to have the \textit{Permit\_CInvoke} permission. The second one is a capability with the \textit{Permit\_Seal} permission. CSeal sets the otype field on the code or data capability to the value in the address field of the second operand. Two capabilities with the same otype value are considered code-data pairs and can be used as input operands of the CInvoke instruction. The CInvoke instruction unseals the code-data pair, moves the code capability to the PCC and places the data capability in a predetermined register.

This system can be used as a security domain transition between two mutually distrusting compartments. The caller cannot tamper with the callee since the references it holds to the callee --- the code-data pair --- are sealed and the callee cannot access any of its caller's state, except for the arguments that the caller passed through CPU registers.

\subsection{Capability Bounds}
\label{sec:bounds}
While the full calculations behind the bounds of a capability are outside of the scope of this thesis, we will explain the general principle behind them in this section. The $I_E$ field is a bit to indicate whether the \textit{E} field is used. When $I_E$ is equal to zero, the size of the \textit{E} field is 0 and $T_E$ and $B_E$ are part of the \textit{T} and \textit{B} fields respectively. If $I_E$ is equal to one, $T_E$ and $B_E$ are joined together to form the \textit{E} field with a size of 6 bits, shortening the \textit{T} and \textit{B} fields in return. These bits that are in this case missing from \textit{T} and \textit{B} are considered to be zeros.

The \textit{B} and \textit{T} fields are used in the calculation of the base and top values --- the addresses which \textit{a} must lie between. The $50 - E$ most significant bits of the base and top are copied from the \textit{a} value, possibly incremented or decremented with a correction value. This correction value is calculated from various comparisons between certain bits of the \textit{T}, \textit{B} and \textit{a} fields. The next 14 bits of the base value are directly copied from the \textit{B} field. The next 14 bits of the top value are copied from the \textit{T} field, extended with 2 bits, based on the \textit{B} and $I_E$ field. The remaining \textit{E} bits are zeros. This means that using the \textit{E} field by setting the $I_E$ bit increases the representable range to potentially the entire address space at the cost of accuracy, introducing alignment requirements. The accuracy cost is inversely proportional to the size of the \textit{E} value.

\subsection{Capability Exceptions}
In general, when an application does something that is not allowed, such as dereferencing a memory location without the proper permissions or modifying a sealed capability, the hardware will raise an exception and will transfer execution to the appropriate exception handler. This allows an operating system to handle the exception in a correct manner.

\subsection{RISC-V Implementation}
\label{sec:cheri-risc-v}
While an implementation of CHERI in MIPS exists, in this thesis we work with the RISC-V implementation of CHERI so we will restrict this section to RISC-V and to the parts of CHERI-RISC-V that are relevant to this thesis. CHERI-RISC-V is an extension of the RISC-V instruction set that implements the CHERI protection model. CHERI-RISC-V uses instructions space that is reserved for extensions in the RISC-V instruction set. This instruction space differs depending on whether compressed RISC-V instructions are used or not. CHERI-RISC-V supports both compressed and non-compressed instructions, but makes some changes for compressed instructions to account for the reduced opcode space.

CHERI-RISC-V supports both a split and a merged register file. With a split register file, registers able to hold capabilities are separate from the integer registers. In contrast, with a merged register file all registers are able to hold both capabilities and integers where the upper 64 bits of a register are ignored if an operand register is interpreted as an integer. A merged register file has significant performance benefits. In this thesis we will work with a merged register file.

Capabilities in CHERI-RISC-V have an extra 1-bit field used for the encoding-mode flag. The addition of this flag reduces the amount of reserved bits from three to two. The encoding-mode flag is used on capabilities that are used as the PCC. When the bit is off, the system is in integer encoding mode, meaning that operands to traditional RISC-V loads and store instructions are interpreted as traditional integer addresses. When the bit is set, the system is in capability encoding mode, meaning that operands to traditional RISC-V load and store instructions are interpreted as capabilities. The big advantage of this mode is that traditional RISC-V load and store instructions can be used with capabilities, saving a significant amount of opcode space.

%%% Local Variables:
%%% mode: latex
%%% TeX-master: "../thesis"
%%% End:

%\section{RISC-V \& SAIL}
RISC-V is an open RISC ISA created by researchers at Berkeley University \cite{Waterman:EECS-2011-62}. Its open nature has made it a popular ISA for education and research purposes with a large amount of available tooling. In this thesis we use CHERI's extension of the RISC-V ISA, called CHERI-RISC-V.

\subsection{Instruction layout}
\label{sec:riscvenc}
RISC-V offers multiple instruction types that differ in the way their bits are interpreted.
In this thesis we will only use R-type instructions, shown in figure \ref{fig:rtypeinst}.

\begin{figure}[h]
\centering
\definecolor{lightgray}{gray}{0.8}
\begin{bytefield}[endianness=big, bitwidth=1em]{32}
    \bitheader{32,25,24,20,19,15,14,12,11,7,6,0} \\
    \bitbox{7}{funct7} & \bitbox{5}{rs2} & \bitbox{5}{rs1} & \bitbox{3}{funct3} & \bitbox{5}{rd} & \bitbox{7}{opcode} \\
\end{bytefield}
\caption{R-type RISC-V instruction.}
\label{fig:rtypeinst}
\end{figure}

An R-type instruction consists of 6 different fields, each with their own meaning. The operation code (\textit{opcode}) field is 7 bits wide and specifies the class of instructions to be used. All of the instructions introduced in this thesis will use the CHERI-specific opcode of \textit{1011011} which is one of the opcodes that RISC-V reserves for extension usage. The exact instruction to execute is further indicated by the \textit{funct3} and \textit{funct7} fields which are 3 and 7 bits wide respectively. The other three fields, \textit{rd}, \textit{rs1} and \textit{rs2} specify which registers should be used in the instruction. \textit{rd} indicates the destination register while \textit{rs1} and \textit{rs2} indicate the first and second source registers.
In CHERI, the \textit{rs2} field is also used as an extra funct field for some instructions, called \textit{funct5}.
This is the case for instructions that use only one source register and a destination register.
We will expand on the usage of this field in section \ref{sec:sailencoding}.
Because CHERI has a distinction between integer and capability registers, CHERI introduces the \textit{cd}, \textit{cs1} and \textit{cs2} notation.
The \textit{c} registers refer to capability registers while \textit{r} registers refer to integer registers.
This distinction is merely cosmetical as the underlying bits just represent a register number and are agnostic to whether the field is interpreted as an integer or capability register.

\subsection{Registers}
The base RISC-V ISA provides 32 integer registers that are named by an x in front of their number.
Reading from the \textit{x0} register always results in reading zero, regardless of what was written to the register.
This means that writing to \textit{x0} has no effect and can be used as a no-operation (NOP).

CHERI-RISC-V introduces another 32 capability registers, named \textit{c0} through \textit{c31} which may or may not be shared with the integer registers.
As mentioned before, in this thesis we work with a shared register file which means that there are only 32 registers and each register can be interpreted as an integer or as a capability register.
Like \textit{x0}, \textit{c0} contains a fixed value which is called the null capability, which has most of its fields set to zero.

%TODO explain opcodes used in our examples?

\subsection{LLVM}
\label{sec:llvm_background}
While several compilers that target RISC-V exist, the only compiler that targets CHERI-RISC-V is CHERI's fork of the LLVM project \cite{cherillvm}.
The LLVM project offers a large amount of tools related to compiling such as a compiler for several programming languages, a linker, an assembler, a disassembler and much more.
In this thesis we will extend CHERI's fork of the LLVM project to support the new instructions we introduce and we will use LLVM's assembler to turn assembly source files into executable binary files.

The code snippet in figure \ref{fig:insttemplate} shows the instruction definition for the CHERI function \textit{CGetPerm} and the template used by that definition.
The template accepts a 5 bit argument \textit{funct5}, a string with the instruction name and an indication whether the used registers are integer or capability register for which defaults are provided.
These arguments are then mapped to an R-type instruction with a fixed \textit{funct7} field of 0x7f, an \textit{rs2} field with the provided \textit{funct5} argument, a fixed \textit{funct3} field of 0, a fixed opcode field with the CHERI-specific opcode of \textit{1011011} and variable \textit{rd} and \textit{rs1} fields depending on the registers that are used for the instruction.
The \textit{CGetPerm} instruction provides the template with a \textit{funct5} field of 0, an instruction string of \textit{cgetperm} and keeps the default register interpretation of an integer destination register and a capability source register.
While other instruction templates exist, they work in a similar way so we won't cover them in detail.

\begin{figure}[h]
\begin{verbatim}
class Cheri_r<bits<5> funct5, string opcodestr, RegisterClass
              rdClass=GPR, RegisterClass rs1Class=GPCR>
    : RVInstCheriSrcDst<0x7f, funct5, 0, OPC_CHERI,
                        (outs rdClass:$rd), (ins rs1Class:$rs1),
                        opcodestr, "$rd, $rs1">;

def CGetPerm : Cheri_r<0x0, "cgetperm">;
\end{verbatim}
\caption{An LLVM instruction template}
\label{fig:insttemplate}
\end{figure}

\subsection{Sail}
Sail is an ISA specification language that can be used to formally describe the semantics of the instructions of an ISA \cite{10.1145/3290384}. Sail models for several different ISA's such as RISC-V, CHERI-RISC-V and ARM have been implemented. SAIL models have a variety of additional uses such as generating documentation, constructing an emulator and formal reasoning about the ISA.
A Sail model consists of a series of instruction definitions, code describing their functionality and a mapping between instruction definitions, binary encodings and mnemonic opcodes.
Because Sail does not offer an assembler, we need to use an other assembler as mentioned before.
In this thesis we will extend the CHERI-RISC-V Sail model and will use the emulator to test a prototype of our design.

The code snippet in figure \ref{fig:sailcode} shows the Sail definition for the \textit{CSetFlags} instruction.
First the instruction name as well as the arguments to the instruction function are described.
Next is the function that describes what happens when \textit{CSetFlags} is executed.
The function definition shows that a capability destination register, a capability source register and an integer source register are used.
The function starts by reading the values in the source registers.
The C(regidx) function can be used to read from or write to the capability register with the specified register index while the X(regidx) function does the same for integer registers.
Then the function inspects some of the fields from the given source capability and throws an exception if those fields contain specific values.
If an exception is not thrown, the function creates a new capability based on the values in the source registers and writes that new capability to the destination register.

\begin{figure}[h]
\begin{verbatim}
union clause ast = CSetFlags : (regidx, regidx, regidx)
function clause execute(CSetFlags(cd, cs1, rs2)) = {
  let cs1_val = C(cs1);
  let rs2_val = X(rs2);
  if cs1_val.tag & isCapSealed(cs1_val) then {
    handle_cheri_reg_exception(CapEx_SealViolation, cs1);
    RETIRE_FAIL
  } else {
    let newCap = setCapFlags(cs1_val, truncate(rs2_val,
                                               cap_flags_width));
    C(cd) = newCap;
    RETIRE_SUCCESS
  }
}
\end{verbatim}
\caption{A snippet of Sail code}
\label{fig:sailcode}
\end{figure}

%%% Local Variables:
%%% mode: latex
%%% TeX-master: "../thesis"
%%% End:

%\input{literature_review/secure_compilation.tex}

%%% Local Variables: 
%%% mode: latex
%%% TeX-master: "thesis"
%%% End: 

%%%% SOME COMMANDS FROM PREVIOUS WORK %%%%
\newcommand{\V}[1]{\ensuremath{\mathit{#1}}}
\newcommand{\X}[1]{\ensuremath{\mathrm{#1}}}
\newcommand{\Sf}[1]{\ensuremath{\mathsf{#1}}}
\newcommand{\I}[1]{\ensuremath{\mathtt{#1}}}
\newcommand{\K}[1]{\ensuremath{\mathsf{#1}}}
\newcommand{\instr}[1]{\texttt{#1}}
\newcommand{\Z}{\mathbb{Z}}
\newcommand{\confv}{\varphi}
\newcommand{\uninitCol}{RoyalBlue}
\newcommand{\locCol}{RedOrange}
\newcommand{\hlnew}[1]{{\color{\uninitCol}#1}}
\newcommand{\hlnewl}[1]{{\color{\locCol}#1}}
% custom emphasise
\newcommand{\emphc}[1]{{\color{red}#1}}
\newcommand{\updc}[1]{#1}
\newcommand{\pc}{\K{pc}}
\newcommand{\rgen}[1]{\K{r_{\X{#1}}}}
\newcommand\RW{\perm{rw}}
\newcommand\RWX{\perm{rwx}}
\newcommand\RWL{\perm{rwl}}
\newcommand\RWLX{\perm{rwlx}}
\newcommand\URW{\perm{urw}}
\newcommand\URWX{\perm{urwx}}
\newcommand\URX{\perm{urx}}
\newcommand\URWL{\perm{urwl}}
\newcommand\URWLX{\perm{urwlx}}
\newcommand\RO{\perm{ro}}
\newcommand\RX{\perm{rx}}
\newcommand\enter{\perm{e}}
\newcommand{\instrsem}[1]{\llbracket #1 \rrbracket}
\newcommand{\permflowsto}{\preccurlyeq}
\newcommand{\permflowsfrom}{\succcurlyeq}
\newcommand{\eqdef}{\triangleq}



\chapter{Designs}
\label{chap:design}
As mentioned in chapter \ref{cha:intro}, the goal of this thesis is to provide support for the Rust safety guarantees on the hardware level as an extension of the CHERI architecture. Several different design directions were explored, each with their own set of tradeoffs to be made. In this chapter we give an overview of the main design and its tradeoffs as well as an argument for its correctness by suggesting how the OBS invariants introduced in section \ref{sec:obsinvariants} can be held. Following that, we give an explanation of the other explored directions along with the reasons these directions were not chosen.

\section{Main Design Overview}
\label{sec:maindesign}
In this design linear capabilities will be used as a starting point. Linear capabilities correspond naturally with Rust's linear ownership system where owners of a value can only be moved, not copied. Implementing Rust owner variables as linear capabilities allows us to use this property to easily guarantee the \textit{unique owner invariant} of the OBS invariants introduced in section \ref{sec:obsinvariants}: if an owner is implemented as a linear capability, it cannot be copied and is thus unique.

Linear capabilities thus represent unique ownership, but can be \emph{borrowed}, transforming them into borrowed capabilities instead, and allowing for linearity to be broken temporarily, and e.g.\ a read-only capability to be shared between multiple functions.
Separate linear tokens, called \emph{lifetime tokens}, define unforgeable lifetimes that linear capabilities are bound to during this borrowing process.
Borrowed capabilities can only be dereferenced while a live (as opposed to dead), matching lifetime token is present in a specific register.
Ending the lifetime corresponds to revocation, meaning that all borrowed capabilities that were bound to the lifetime are now unable to be used.
To mirror Rust, we provide for two types of borrowing; mutable and immutable borrowing, depending on whether the capability requires write access or not.

To describe our design more precisely, we define some basic syntactic constructs for our capability machine in fig.\ \ref{fig:opsem_syntax}.
The top section defines pre-existing notions for capabilities and their fields while the middle section defines the constructs we add which we will explain throughout this section.
The bottom section lists the state of our machine, consisting of memory and registers, and lists the new instructions we add.
The semantics of the new instructions are shown in fig.\ \ref{fig:opsem_instrs}.

\begin{figure}
\arraycolsep=3pt
\[
\begin{array}{lllll}
  %%% Machine words
  a & \in & \X{Addr} & \eqdef & [0, \X{AddrMax}] \\
  p & \in & \X{Perm} & \eqdef & \perm{o} \mid \perm{ro} \mid \perm{rx} \mid \perm{rw} \mid \perm{rwx}\\
  o & \in & \X{Otype} & \eqdef & [0, \X{OtypeMax}] \\
  l & \in & \X{Linear} & \eqdef & 0 \mid 1 \\
  c & \in & \X{Cap} & \eqdef & \{(p, o, l, b, e, a) \mid b, e, a \in \X{Addr}\} \\[.3em] \hline\rule{0pt}{2.6ex}
  s & \in & \X{State} & \eqdef & \perm{a} \mid \perm{d}\\
  f & \in & \X{Fraction} & \eqdef & [0, \X{FractionMax}]\\
  \V{lt} & \in & \X{LifetimeToken} & \eqdef & \{\X{lt}(s, lid, cid, pid, f) \mid lid, cid, pid \in \X{Otype}\} \\
  idx & \in & \X{Index} & \eqdef & [0, \X{IndexMax}]\\
  \V{it} & \in & \X{IndexToken} & \eqdef & \left\{ \X{it}(lid, idx) \mid lid \in \X{Otype} \right\}\\
  bt & \in & \X{BorrowTable} & \eqdef & \X{Index} \rightharpoonup \X{Cap} \\[.3em] \hline\rule{0pt}{2.6ex}
  % w & \in & \X{Word} & \eqdef & \Z + \X{Cap} \\
  r & \in & \X{RegName} \\%& \eqdef & \rgen{0} \mid \rgen{1} \mid \ldots \mid \rgen{31} \\
  % \V{reg} & \in & \X{Reg} & \eqdef & \X{RegName} \rightarrow \X{Word} \\
  \confv & \in & \X{ExecConf} & \eqdef & \X{Reg} \times \X{Mem}
\end{array}%
\]%
%
\[
\begin{array}{lll}
%%% Instructions
  i & \mathrel{::=} &\ldots \mid \instr{CCreateToken}\; r \; r \mid
               \instr{CKillToken}\; r \; r \mid
               \instr{CUnlockToken}\; r\; r\; r \mid \\
               &&\instr{CSplitLT}\; r\; r \mid
               \instr{CMergeLT}\; r \; r \; r \mid
               \instr{CBorrowImmut}\; r \; r \; r \mid \\
               &&\instr{CBorrowMut}\; r \; r \; r \mid
               \instr{CRetrieveIndex}\; r \; r \; r
\end{array}
\]
\caption{\label{fig:opsem_syntax}Machine words, machine state and added instructions.}
%\vspace{-0.5cm}
\end{figure}

\begin{figure}[ht]
  \small
  \center

\begin{tabular}{|c|l|l|}
  \hline
  $i$ & $\instrsem{i}(\confv)$ & Conditions \\ \hline
  %
  \begin{tabular}{c}
  $\instr{CCreateToken}\; $ \\ $ r_1\; r_2$
  \end{tabular}
      & \begin{tabular}{l}
   $\confv[\X{reg}.r_1 \mapsto w_1,$\\ \quad $ \X{reg}.r_2 \mapsto w_2]$
        \end{tabular}
  & \begin{tabular}{l}
      $\text{if }\confv.\X{reg}(r_2) == \X{lt}(D,0,0,0,0)\text{ then } $\\
      \ \ \ $w_1 = \X{lt}(A,lid,0,0,0)$ and \\
      $w_2 = \X{lt}(D,0,0,0,0)$ \\
      $\text{else if }\confv.\X{reg}(r_2) ==$ \\ $\X{lt}(A,pid,0,ppid,f)\text{ then }$ \\
      \ \ \ $w_1 = \X{lt}(A,lid,0,pid,0)$ and \\
      \ \ \ $w_2 = \X{lt}(A,pid,lid,ppid,f)$
    \end{tabular}
  \\ \hline
  %
    \begin{tabular}{c}
  $\instr{CKillToken}\;$ \\ $ r_1\; r_2$
    \end{tabular}
      & \begin{tabular}{l}
   $\confv[\X{reg}.r_1 \mapsto w_1,$\\ \quad $ \X{reg}.r_2 \mapsto 0]$
        \end{tabular}
  & \begin{tabular}{l}
      $\confv.\X{reg}(r_2) = \X{lt}(A,lid,0,pid,f_{max})$ and \\
      $w_1 = \X{lt}(D,lid,0,pid,f_{max})$
    \end{tabular}
  \\ \hline
  %
    \begin{tabular}{c}
  $\instr{CUnlockToken}\;$ \\ $ r_1\; r_2\; r_3$
    \end{tabular}
      & \begin{tabular}{l}
   $\confv[\X{reg}.r_1 \mapsto w_1,$\\ \quad $ \X{reg}.r_2 \mapsto 0]$
        \end{tabular}
  & \begin{tabular}{l}
      $\confv.\X{reg}(r_2) = \X{lt}(A,lid,cid,pid,f)$ and \\
      $\confv.\X{reg}(r_3) = \X{lt}(D,cid,\_,\_,\_)$ and \\
      $w_1 = \X{lt}(A,lid,0,pid,f)$
    \end{tabular}
  \\ \hline
  %
    \begin{tabular}{c}
  $\instr{CSplitLT}\;$ \\ $ r_1\; r_2$
    \end{tabular}
    & \begin{tabular}{l}
   $\confv[\X{reg}.r_1 \mapsto w_1,$\\ \quad $ \X{reg}.r_2 \mapsto w_2]$
      \end{tabular}
  & \begin{tabular}{l}
      $\confv.\X{reg}(r_1) = \X{lt}(A,lid,cid,pid,f)$ and \\
      $w_1 = \X{lt}(A,lid,cid,pid,f + 1)$ and \\
      $w_2 = \X{lt}(A,lid,cid,pid,f + 1)$
    \end{tabular}
  \\ \hline
  %
    \begin{tabular}{c}
  $\instr{CMergeLT}\;$ \\ $ r_1\; r_2\; r_3$
    \end{tabular}
      & \begin{tabular}{l}
          $\confv[\X{reg}.r_1 \mapsto w_1,$\\ \quad $ \X{reg}.r_2 \mapsto 0,$\\
          \quad $\X{reg}.r_3 \mapsto 0]$
    \end{tabular}
  & \begin{tabular}{l}
      $\confv.\X{reg}(r_2) = \X{lt}(A,lid,cid,pid,f)$ and \\
      $\confv.\X{reg}(r_3) = \X{lt}(A,lid,cid,pid,f)$ and \\
      $w_1 = \X{lt}(A,lid,cid,pid,f - 1)$
    \end{tabular}
  \\ \hline
  %
    \begin{tabular}{c}
  $\instr{CBorrowImmut}\;$ \\ $ r_1\; r_2\; r_3$
    \end{tabular}
  & \begin{tabular}{l}
      $\confv[\X{reg}.r_1 \mapsto w_1,$\\ \quad $ \X{reg}.r_2 \mapsto w_2,$ \\
      \quad$\X{bt}.idx \mapsto $\\ \quad $ \confv.\X{reg}(r_2)]$
  \end{tabular}
  & \begin{tabular}{l}
      $\confv.\X{reg}(r_2) = (p,o,1,b,e,a)$ and \\
      $\confv.\X{reg}(r_3) = \X{lt}(A,lid,\_,pid,\_)$ and \\
      $\text{if }o == 0\ or\ o == pid\text{ then } $\\
      \ \ \ $w_1 = \X{it}(lid, idx)$ and \\ $w_2 = (\perm{RO},lid,0,b,e,a)$
    \end{tabular}
  \\ \hline
  %
    \begin{tabular}{c}
  $\instr{CBorrowMut}\;$ \\ $ r_1\; r_2\; r_3$
    \end{tabular}
  & \begin{tabular}{l}
      $\confv[\X{reg}.r_1 \mapsto w_1,$\\ \quad $ \X{reg}.r_2 \mapsto w_2,$ \\
      \quad$\X{bt}.idx \mapsto $\\ \quad $ \confv.\X{reg}(r_2)]$
    \end{tabular}
  & \begin{tabular}{l}
      $\confv.\X{reg}(r_2) = (p,o,1,b,e,a)$ and \\
      $\confv.\X{reg}(r_3) = \X{lt}(A,lid,\_,pid,\_)$ and \\
      $\text{if }o == 0\ or\ o == pid\text{ then } $\\
      \ \ \ $w_1 = \X{it}(lid, idx)$ and \\ $w_2 = (\perm{RW},lid,1,b,e,a)$
    \end{tabular}
  \\ \hline
  %
    \begin{tabular}{c}
  $\instr{CRetrieveIndex}\;$ \\ $ r_1\; r_2\; r_3$
    \end{tabular}
  & \begin{tabular}{l}
      $\confv[\X{reg}.r_1 \mapsto w_1,$\\ \quad $ \X{reg}.r_2 \mapsto 0,$\\ \quad $ \X{bt}.idx \mapsto 0]$
      \end{tabular}
  & \begin{tabular}{l}
      $\confv.\X{reg}(r_2) = \X{it}(lid, idx)$ and \\
      $\confv.\X{reg}(r_3) = \X{lt}(D,lid,\_,\_,\_)$ and \\
      $w_1 = \confv.\X{bt}(idx)$
    \end{tabular}
  \\ \hline
\end{tabular}
\caption{\label{fig:opsem_instrs}Operational semantics for essential cases in our novel instructions.}
%\vspace{-0.5cm}
\end{figure}

We represent the fields of lifetime tokens as a tuple of the form $\X{lt}(s,\V{lid},\V{cid},\V{pid},f)$ representing the state, lifetime id, child id, parent id and fraction fields.
The state field indicates whether the lifetime associated with the lifetime token is alive (\perm{A}) or dead (\perm{D}).
The first case of the \instr{CCreateToken} instruction in fig.\ \ref{fig:opsem_instrs} creates a normal standalone alive lifetime token.
The second case will be explained later in this section.
Alive lifetime tokens are linear and can be used to borrow new capabilities, and dereference existing borrowed capabilities when placed in a specific register.
A lifetime token can be irreversibly killed with the \instr{CKillToken} instruction.
This changes the state to dead which results in the lifetime token losing its linearity.
In this state, a lifetime token cannot be used anymore to borrow capabilities or dereference borrowed capabilities, but it does serve as a proof of the lifetime's end that can be freely copied and passed around.
The lifetime id field holds the lifetime id associated with the token.
This is a unique id that is used to match a borrowed capability with its corresponding lifetime token.
The fraction, parent id and child id fields will be discussed later in this section.

A lifetime token can be used to borrow a capability through the \instr{CBorrowImmut} or \instr{CBorrowMut} instructions, for immutable and mutable borrows respectively.
What happens in these borrow operations is that the capability that gets borrowed is stored in a table that we call the \emph{borrow table} ($bt$).
The source register is overwritten with the borrowed capability.
This borrowed capability points to the same region of memory as the original capability but it differs in a few ways.
First, the borrowed capability holds the lifetime id of the lifetime it was borrowed under in its otype field.
This is necessary in order to check whether the used lifetime token matches the borrowed capability when dereferencing it.
Second, depending on whether the original capability was immutably or mutably borrowed, the borrowed capability might have differing permissions.
Immutably borrowed capabilities lose their linearity as well as their write permissions if they were present on the original capability.
This weakens the original linear capability's exclusive access to a resource, but does so in a controlled manner, namely for the duration of the lifetime.
This is sufficient to prevent simultaneous write accesses while allowing multiple references with read access.
This behavior allows Rust to be mapped to borrowed capabilities.
Mutably borrowed capabilities keep their linearity and write access like in Rust.

As previously mentioned, when a capability is borrowed, the original capability is stored in the borrow table, to be retrieved after the lifetime ends.
In order to retrieve capabilities from the borrow table, we introduce linear \emph{index tokens} that can be traded for the original capability in the borrow table through the \instr{CRetrieveToken} instruction.
We represent index tokens as a tuple of the form $\X{it}(\V{lid},\V{idx})$, keeping track of the lifetime id $\V{lid}$ under which the capability stored at index \V{idx} was borrowed.
Index tokens are produced by the previously described borrow instructions.
The \instr{CRetrieveToken} instruction requires both the index token as well as a dead corresponding lifetime token that acts as proof that the borrowed capabilities cannot be used anymore.

To allow dereferencing immutable borrows in multiple threads concurrently like Rust allows, we allow lifetime tokens to have a fraction $f$ and be split using the \instr{CSplitToken} instruction.
Fractions are represented as an integer where a fraction of 0 signifies the full lifetime token.
Splitting a lifetime token with fraction $f$ results in two copies of the lifetime token with respective fractions $f_1$ and $f_2$, such that $f_1 = f_2 = f + 1$.
Both fractions can still be used to borrow capabilities or dereference borrowed capabilities.
Merging identical lifetime tokens is possible with the \instr{CMergeToken} instruction.
% Splitting a lifetime token increments the value in the fraction field on both resulting tokens, while merging a lifetime token decrements the value in the fraction field.
To prevent dead and alive lifetime tokens of the same lifetime id being present at the same time, only unfractured lifetime tokens can be killed.
% The fraction system allows software to split a lifetime token and distribute the fractions and corresponding (immutably) borrowed capabilities to different threads for simultaneous read access.
Once different threads using the fractions of the lifetime token have completed their work, they can return the lifetime fractions which can then be merged again to the full lifetime token, which can then be killed.

Care must be taken when borrowing borrowed capabilities, a reborrow operation like we explained in section \ref{sec:backgroundreborrow}.
Borrowed capabilities cannot be allowed to be reborrowed under just any lifetime.
This would make it possible to reborrow a capability under a lifetime that is longer than the original borrow which breaks the Rust guarantees.
To make reborrows possible, we introduce lifetime hierarchies.
With this system, lifetimes can be created as sublifetimes of existing lifetimes by providing a fraction of the desired parent lifetime to the \instr{CCreateToken} operation as shown in the second case in figure \ref{fig:opsem_instrs}.
This sets the child id field on the parent to the lifetime id of the newly created lifetime and sets the parent id field of the newly created lifetime token to the lifetime id of the parent.
Lifetime tokens cannot be killed while they have a child id set, but don't lose any of their other functionality.
This ensures that sublifetimes cannot last longer than their parent lifetime while still keeping the parent lifetime available.
In order to remove a child from a parent lifetime token, a dead lifetime token with the lifetime id of the child id field on the parent token is needed.
These two tokens can then be used in the \instr{CUnlockToken} instruction which clears the child id field on the parent token.
The parent token can then be killed or receive a new sublifetime.

Lifetime hierarchies allow for safe reborrowing through the normal borrow operations \instr{CBorrowImmut} and \instr{CBorrowMut}.
The main requirement for reborrowing borrowed capabilities is that the parent id on the used lifetime token matches the lifetime id of the capability that is being reborrowed.
This ensures that a reborrow happens under a sublifetime and thus that the reborrow has a shorter lifetime than the original borrow.
One issue with lifetime hierarchies is that a lifetime token can only have one parent, which makes it impossible to reborrow multiple capabilities with different lifetimes under the same lifetime, something that is allowed in Rust.
We will address this further in section \ref{sec:dynamiclifetimes}.

\subsection{Advantages}
The main advantage of this design is that it mimicks the Rust semantics fairly closely and that it is relatively easy to fit into the CHERI architecture. While some fields of the CHERI capability layout have to be partially repurposed, this design can be implemented without using additional bits and without making deep changes to the CHERI architecture.

\subsection{Disadvantages}
\label{sec:designdisadvantages}
Issues with the design pertain to differences with Rust semantics. In Rust when an owner is immutably borrowed, the owner can still be used to read, however, this is impossible if the owner is locked away. A solution would be to borrow the owner for local use and then reborrow under a sublifetime. Another issue is that in this design reborrows can only occur under direct lifetime children while in Rust any lifetime that is contained within another lifetime can be used to reborrow. This can also be solved by inserting extra borrows, at the cost of performance.

Another concern is the presence of operations that require writing to multiple registers which might be hard to implement in microarchitecture. We keep this limitation in mind, but do not attempt to mitigate it.

\subsection{OBS Invariants}
\begin{enumerate}
    \item \textit{Unique owner invariant}: holds as a result of owner capabilities being linear.
    \item \textit{Lifetime inclusion invariant}: holds as a result of capabilities not being able to be retrieved from the borrow table unless a dead lifetime token is presented.
    \item \textit{Lifetime disjoint invariant}: holds as a result of capabilities being locked away in the hardware table, preventing any further borrows.
    \item \textit{Writing permission invariant}: holds as a result of capabilities being locked away in the hardware table, preventing any dereferences.
    \item \textit{Reading and writing permissions invariant}: holds as a result of capabilities being locked away in the hardware table, preventing any dereferences.
\end{enumerate}

\section{Variations on the Current Design}
In this section we explain some variations on the main design that was discussed above.

\subsection{Dynamic Lifetimes}
\label{sec:dynamiclifetimes}
At the end of section \ref{sec:maindesign} we mentioned a problem with lifetime hierarchies in that lifetime tokens are unable to have multiple parent lifetimes.
A solution to this that we considered, but did not implement due to time constraints is called \textit{dynamic lifetimes}.
This involves borrowing a fraction of the parent lifetime under the child lifetime, and thereby storing this fraction in the borrow table.
Since the fraction of the parent that is needed to kill it can only be retrieved from the borrow table with the help of the dead child lifetime token, it is ensured that the parents' lifetime is longer than the child's.
This scheme allows any lifetime token to dynamically become a child of another lifetime token.
Reborrowing a capability with the parent's lifetime under the child's lifetime would then be possible by providing the borrow instruction with the borrowed lifetime fraction, as dynamic proof of relationship between the parent and child lifetime.
Dynamic lifetimes would make the lifetime system more flexible at the cost of extra stores to the borrow table, and the complexity of managing the index tokens for borrowed lifetime token fractions.

\subsection{Key Tokens}
In our design we use the borrow table and index tokens to store and retrieve capabilities that temporarily need to lose access to a resource.
An alternative to this scheme that we considered uses another type of separate token called \textit{key tokens}.
Instead of storing capabilities in a borrow table, this scheme would simply reduce permissions on them.
Because they need to regain those permissions after a borrow, the permissions are stored in the key token that inactive by default.
A dead lifetime token can be used to make the key token active, allowing it to be used to restore the permissions on the original capability.
The advantage of this scheme is that the original capability remains present and if it was for example immutably borrowed, it can still be used to read.
The big issue with actually implementing this design is that key tokens need to be uniquely linked to one specific capability which requires extra bits on the capability.
For normal borrows the otype field could be used to store some sort of id, but this would not work for reborrows as the otype field is already being used for the lifetime id.
In the end, while this design has a lot of upsides, linking key tokens with capabilities is complicated and the alternative of a borrow table with index tokens is a lot more simple.

\section{Different Designs}
In this section we explain explored designs that are completely different from the main design.

\subsection{Tree Structure with Reference Counting}
In this design, every capability has its own unique id and a reference to its parent's id where a parent id of zero signifies an owner and any other id signifies a borrowed capability.
All capabilities have a field that indicates whether the capability was borrowed mutable and a reference counting field that indicates how many times they were immutably borrowed.
Depending on the values in these fields, a capability loses certain permissions such as write or read access or the ability to be borrowed again.
The fields on a parent can be altered by returning borrowed capabilities to their parent which would remove the mutably borrowed field or decrement the immutable borrow reference count.
An advantage of this scheme is that it supports reborrows inherently and is conceptually fairly simple.
The big disadvantages are the large amount of bits required for the id's and reference counts and the fact that immutably borrowed capabilities need to be linear to preserve the correctness of the reference count on the parent.
An interesting thing to note is that lifetime hierarchies show parallels with this design and move parts of it to a separate token.

\subsection{Indirection}
One design that we did not think about in great depth uses indirection for borrowed capabilities.
In this scheme, borrowed or reborrowed capabilities would point to their parent capability, either in memory or in a separate table.
Borrowed capabilities and their parents would hold a unique id which would need to match in order to follow the chain of indirection.
Changing the id on a capability would correspond to ending the lifetime on all of its borrowed capabilities.
This obvious flaws with this design are the performance impact and the lack of flexibility since the chain of indirections needs to be upheld.
It also differs from Rust semantics in a few ways.

\chapter{Mapping Borrowed Capabilities to Architecture}
In this chapter we will discuss the concrete implementation of the design outlined in section \ref{sec:maindesign}. First, we will discuss the specifics of linear capabilities which are based on previous work. We lean heavily on the suggestions outlined in the CHERI specification\cite{UCAM-CL-TR-951} for the implementation of linear capabilities. Then we describe the implementation of borrowed capabilities which are built on top of linear capabilities. Throughout this chapter we keep possible microarchitectural limitations in mind, but we do not discuss them in depth.

\section{Linear Capabilities}
Figure \ref{fig:linear_capability} shows the layout of a 128-bit CHERI-RISC-V capability with linearity support. The only differences between this format and the format shown in \ref{fig:cheri_capability} is the addition of a \textit{flag} field, required in the CHERI-RISC-V architecture as described in section \ref{sec:cheri-risc-v} and the addition of a \textit{linear} field. The linear field consists of only one bit which indicates whether the capability is linear or not.

\begin{figure}[h]
\centering
\definecolor{lightgray}{gray}{0.8}
\begin{bytefield}[endianness=big, bitwidth=.55em]{64}
    \bitheader{0,63} \\
    \bitbox{13}{\textit{p}'16} & \bitbox{2}{\color{lightgray}\rule{\width}{\height}} & \bitbox{2}{\textit{l}} & \bitbox{2}{\textit{f}} & \bitbox{15}{otype'18} & \bitbox{3}{\textit{$I_E$}} & \bitbox{8}{\textit{T}[11:3]} & \bitbox{5}{\textit{$T_E$}'3} & \bitbox{9}{\textit{B}[13:3]} & \bitbox{5}{\textit{$B_E$}'3} \\
    \bitbox{64}{\textit{a}'64}
\end{bytefield}
\caption{CHERI-RISC-V capability with a linearity bit.}
\label{fig:linear_capability}
\end{figure}

When the \textit{linear} bit on a capability is set, it cannot be copied in any way. This restriction has influence on a large amount of instructions in the ISA. We describe the necessary modifications to existing instructions as well as the new instructions that were made in the sections below.

\subsection{Instructions with Modification Semantics}
The first set of instructions are a class of instructions that modify existing capabilities. These instructions are allowed to have a destination register that is different from their capability source register. This is analogous to most normal RISC-V instructions where the destination register can also be different from the source. This design makes sense in most cases, but it poses difficulties for linear capabilities. The most obvious linearity violation is an update to a capability that does not even modify the capability. A simple example of this would be the removal of a permission that was not present on the source capability. This results in the capability being copied unmodified from the source to the destination register which is an obvious violation of linearity.

However, even modifications to a capability usually constitute a linearity violation. When using capabilities as pointers to a section of memory, linearity is usually used as a means of proving exclusive access to that section of memory. Any instruction that produces a modified capability that points to the same memory section as the source capability or a subset of it can potentially violate linearity.

To prevent these violations of linearity, we have chosen to modify the existing CHERI instructions with modification semantics to raise a hardware exception when they try to output the modified capability to a different register than the register of the source capability if the source capability is linear. This prevents capabilities from being copied as the source capability will always be overwritten by the modified capability.It is the responsibility of the compiler to make sure that the destination register equals the source register when trying to modify a linear capability.

This is a fairly unintrusive change since it only modifies the instructions' behavior when the linearity bit is set which means it is backwards compatible with existing CHERI software. This change should also be fairly straightforward to implement in microarchitecture. This approach does have a minor impact on performance as an extra move instruction for linear capabilities might be necessary in some cases. Another concern is the effect on compiler code generation since due to this change, more instructions may trap for more reasons than before. This might make it harder for the compiler to generate efficient code.

To go along with the changes to the instructions, an extra exception type named \textit{CapEx\_LinearityViolation} was added to the list of hardware exceptions.

Table \ref{table:linearitymodification} contains the existing CHERI instructions with modification semantics that had to be modified.

\begin{table}[h]
\centering
\begin{tabular}{| c |}
\hline
 CAndPerm \\
 CSetFlags \\
 CIncOffset \\
 CIncOffsetImmediate \\
 CSetOffset \\
 CSetAddr \\
 CSetBounds \\
 CSetBoundsImmediate \\
 CSetBoundsExact \\
 CFromPtr \\
 CBuildCap \\
 CCopyType \\
 CSeal \\
 CCSeal \\
 CUnseal \\
 CSealEntry \\
\hline
\end{tabular}
\caption{Modification Instructions}
\label{table:linearitymodification}
\end{table}

\subsection{Instructions with Move Semantics}
A second set of instructions consists of instructions that are meant to move capabilities between (special) registers. Using the approach from the previous section where the destination register must equal the source register is not an option here as this would defeat the purpose of these instructions.

The most simple of these instructions is the \textit{CMove} instruction which simply moves a capability from the source register to the destination register. This move is essentially a copy since the instruction does not clear the source register. To prevent this violation of linearity we have opted to clear the tag of the source register if the capability that is being moved is linear. This invalidates the capability in the source register. While clearing the source register seems like a straightforward and simple operation, this requires writing to multiple registers in one instruction and might pose a microarchitectural challenge. The other instructions described in this section use the same solution.

The \textit{CJALR} instruction copies the capability in the PCC special register to the destination register and replaces it with the capability in the source register. If the source capability is linear, the tag on the source register will be cleared.

The \textit{CInvoke} instruction described in section \ref{sec:sealed} places the capability in the first source register in the PCC and moves the capability in the second source register to \textit{C31}. Both source registers for this instruction have the opportunity to violate linearity. For the first source capability, this is solved like the other instructions in this section: if it is linear, the tag of the first source register will be cleared. For the second source capability the linearity violation is prevented by raising a hardware exception if the second source capability is linear and the second source register is not equal to \textit{C31}. This is analogous to the instructions in the previous section.

The \textit{CSpecialRW} instruction copies the capability in the specified special register to the destination register and replaces it with the capability in the source register. If the capability in the source register is equal to \textit{C0}, the special register is not replaced. To make this instruction comply with the linearity bit, it was changed to clear the tag of the special register if the capability in it is linear and if the source register is equal to \textit{C0}. However, if the source register is not equal to \textit{C0}, this becomes unnecessary as the special register is overwritten anyways. In this case there is a linearity violation when the source register has its \textit{linear} bit set. Thus the instruction was changed to raise a hardware exception if the capability in the source register is linear and the destination register does not equal the source register.

The \textit{AUIPCC} instruction copies the capability in the PCC to a regular register. For this instruction we chose to clear the tag on the destination register if the capability in the PCC has it \textit{linear} bit set and to not modify the capability in the PCC. This is consistent with CHERI's design of not letting software manipulate the PCC directly.

\subsection{Load \& Store Instructions}
A third set of instructions that need to be modified are load and store instructions, specifically those dedicated to loading and storing capabilities. We follow the suggestion in the CHERI specification and introduce new load and store instructions to load and store linear capabilities as well as modify existing load and store instructions to respect linearity. Table \ref{table:loadstoreinst} contains the modified and added instructions.

\begin{table}[h]
\centering
\begin{tabular}{| c | c | c | c |}
\hline
 Existing Load & Existing Store & New Load & New Store \\
 \hline
 LoadCapDDC & StoreCapDDC & & \\
 LoadCapImm & StoreCapImm & & \\
 LoadCapCap & StoreCapCap & LinearLoadCapCap & LinearStoreCapCap \\
\hline
\end{tabular}
\caption{Load \& Store Instructions}
\label{table:loadstoreinst}
\end{table}

Load instructions were modified to clear the tag of the destination register if the loaded capability has its \textit{linear} bit set. This means that linear capabilities cannot be loaded from memory through these instructions as they will always become invalid. This approach was taken to avoid having to write to memory depending on the capability that was loaded, simplifying the microarchitectural implementation.

Store instructions were modified to clear the tag of the source register if the stored capability has its \textit{linear} bit set. This means that it is possible to successfully store linear capabilities through regular store instructions, in contrast to load instructions. The reason for this design is that clearing the tag on a register conditioned on the capability in that register is much cheaper than clearing a tag in memory conditioned on a loaded capability as is the case with load instructions.

Because linear capabilities cannot be successfully loaded from memory through regular load instructions, a new linear load instruction is necessary. This is the newly introduced \textit{LinearLoadCapCap} which loads a capability from memory to the destination register based on the \textit{address} field of the capability in the source register. Loading a capability in this manner will always clear the tag in memory of the loaded capability, regardless of whether it has its \textit{linear} bit set. This unconditional write avoids the issue with conditional writes that was encountered with the regular load instructions. As seen in table \ref{table:loadstoreinst}, we chose to only introduce an instruction that loads capabilities based on the \textit{address} field of a capability as opposed to including instructions that load capabilities based on the DCC or an immediate. This decision was made to save on instruction encoding space, but might cause a performance impact.

The new \textit{LinearStoreCapCap} instruction is the store equivalent of the \textit{LinearLoadCapCap} instruction. It stores the capability in the first source register in memory based on the \textit{address} field of the capability in the second source register and unconditionally clears the tag of the first source register. This instruction is not strictly necessary as linear capabilities can be successfully stored by regular store instructions, but the presence of this instructions adds symmetry between linear load and store instructions.

\subsection{Linearity Instructions}
Finally, two new instructions were added to inspect and manipulate a capability's \textit{linear} bit and two new instructions were added to split and merge linear capabilities. These split and merge instructions will also be useful for borrowed capabilies as shown in section \ref{sec:borrowmodinsts}. The added instructions are shown in table \ref{table:linearityinst}.

The \textit{CGetLinear} instruction sets the least significant bit of the integer in the destination register to the same value as the \textit{linear} bit of the capability in the source register and clears all the other bits of the integer. This behavior is analogous to the already existing capability field inspection instructions such as \textit{CGetPerm}.

The \textit{CMakeLinear} instruction copies the capability in the source register to the destination register and sets its \textit{linear} bit. This irreversibly\footnote{As part of the design of borrowed capabilities we will provide a controlled way to break linearity for some capabilities.} makes a capability linear to ensure that it cannot be copied. Of course a capability can be copied before it is made linear. This allows software to delegate a linear capability to an untrusted component to restrict its use of the capability while retaining an unrestricted copy of the capability itself. There may be applications of linear capabilities that require unsetting or overriding the \textit{linear} bit on a capability, but for the purposes of this thesis, the irreversible \textit{CMakeLinear} operation is sufficient.

The \textit{CSplitCap} instruction splits the capability in the first source register in two with the boundaries of both parts determined by the integer offset specified in the second source register. After the execution of this instruction, the top of the capability in the first source register is changed to its base plus the specified offset and the destination register holds a copy of the capability in the first source register with its base set to the original base plus the specified offset plus one. This instruction is intended to split linear capabilities but can also be used on other capabilities. Because this instruction writes to multiple registers, it might be hard to implement in microarchitecture.

The \textit{CMergeCap} instruction reverts the split from \textit{CSplitCap}. It requires two capabilities pointing to contiguous memory sections in its source registers and joins them together to form a single capability that points to the entire memory section. This instruction clears the tags on its source registers. Like \textit{CSplitCap}, this instruction is intended to be used with linear capabilities, but can also be used on other capabilities.

\begin{table}[h]
\centering
\begin{tabular}{| c |}
\hline
 CGetLinear \\
 CMakeLinear \\
 CSplitCap \\
 CMergeCap \\
\hline
\end{tabular}
\caption{Linearity Instructions}
\label{table:linearityinst}
\end{table}

\section{Borrowed Capabilities}
In this section we discuss the implementation of the design of borrowed capabilities. We start off with a look at the memory representation of the newly introduced tokens. Next, we describe the newly required architectural elements and finally we expand on the changes needed to existing instructions as well as the definitions of new instructions.
\subsection{Token Layouts}
In this section we introduce the memory layout of the lifetime and index tokens and discuss the design decisions pertaining to these tokens. It was decided to fit all of the necessary new fields into fields that also exist in the regular CHERI capability layout to simplify the interpretation of fields. This means that all of the new fields are either renamed from an existing field or fit into a larger existing field. We do not combine bits of two separate existing fields into one new field.

\subsubsection{Lifetime Tokens}
Figure \ref{fig:lifetime_token} shows the layout of a lifetime token.

\begin{figure}[h]
\centering
\definecolor{lightgray}{gray}{0.8}
\begin{bytefield}[endianness=big, bitwidth=.55em]{64}
    \bitheader{0,63} \\
    \bitbox{13}{\textit{p}'16} & \bitbox{2}{\color{lightgray}\rule{\width}{\height}} & \bitbox{2}{\textit{l}} & \bitbox{2}{\textit{f}} & \bitbox{15}{otype'18} & \bitbox{3}{\textit{$I_E$}} & \bitbox{13}{\textit{T}'11} & \bitbox{14}{fraction'13} \\
    \bitbox{10}{\color{lightgray}\rule{\width}{\height}} & \bitbox{18}{parent id'18} & \bitbox{18}{child id'18} & \bitbox{18}{lifetime id'18}
\end{bytefield}
\caption{The layout of a lifetime token.}
\label{fig:lifetime_token}
\end{figure}

The \textit{permissions} field is unaltered from the regular capability layout, but since none of the permissions are relevant for lifetime tokens, all permissions are set to zero.

The \textit{linear} field serves the same purpose as it does in linear capabilities, but it also indicates whether a lifetime token is alive or dead. Alive lifetime tokens always have their \textit{linear} bit set, while dead lifetime tokens never do.

The \textit{flags} field is unaltered from the regular capability layout.

The \textit{otype} field is always set to the identifier for lifetime tokens: $2^{18} - 3$. This allows both the software and the hardware to confirm whether a capability is a lifetime token by checking its \textit{otype} field.

The $I_E$ bit is always set to 0. This allows us the use the full width of the \textit{B} field which is repurposed to the \textit{fraction} field. The \textit{fraction} field holds the fraction of the lifetime token. A full lifetime token has a fraction of zero. When a full lifetime token is split, two lifetime tokens with fraction 1 are produced. Those lifetime tokens can then be split again, producing lifetime tokens with fraction 2. Lifetime tokens can continue to be split until the maximum value of of the \textit{fraction} field is reached. Since the \textit{fraction} field is 13 bits wide, the theoretical maximum amount of lifetime tokens is $2^{13}=8192$. Two lifetime tokens can be merged if their fractions are equal and not zero. The merging of two identical lifetime tokens results in a lifetime token with its \textit{fraction} field decremented.

The \textit{T} field is not used in lifetime tokens and is set to zero.

The previous \textit{a} field is split up into three different fields to hold a lifetime token's own lifetime id, the id of its parent and the id of its child. Because each of these fields is 18 bits wide, the bottom word of a capability contains 10 unassigned bits. While the necessity of a lifetime token holding its own id is clear, the need for fields to store the parent and child id's might be less obvious. Reborrowing a capability under a new lifetime requires the parent id of the new lifetime to be the lifetime id that the capability was originally borrowed under. This can only be checked if a lifetime token holds the id of its parent. The necessity of the \textit{child id} field arises from the restriction that lifetime tokens cannot be killed while they have a child. This means that there needs to be a way to determine whether a lifetime token has a child as well as a way to modify the lifetime token when its child lifetime is dead. It might seem possible to use a single bit to indicate whether a specified lifetime token has a child and clear that bit through an operation that uses a dead lifetime token with the \textit{parent id} field set to the lifetime id of the specified lifetime token. However, since lifetime tokens can have multiple children in succession, using one bit to track whether a lifetime token has a child is unsufficient as this would allow old children to be used to remove the bit. Storing the entire lifetime id of the child lifetime in the \textit{child id} field solves this problem.

\subsubsection{Index Tokens}
Figure \ref{fig:index_token} shows the layout of an index token.

\begin{figure}[h]
\centering
\definecolor{lightgray}{gray}{0.8}
\begin{bytefield}[endianness=big, bitwidth=.55em]{64}
    \bitheader{0,63} \\
    \bitbox{13}{\textit{p}'16} & \bitbox{2}{\color{lightgray}\rule{\width}{\height}} & \bitbox{2}{\textit{l}} & \bitbox{2}{\textit{f}} & \bitbox{15}{otype'18} & \bitbox{3}{\textit{$I_E$}} & \bitbox{8}{\textit{T}[11:3]} & \bitbox{5}{\textit{$T_E$}'3} & \bitbox{9}{\textit{B}[13:3]} & \bitbox{5}{\textit{$B_E$}'3} \\
    \bitbox{30}{\color{lightgray}\rule{\width}{\height}} & \bitbox{18}{lifetime id'18} & \bitbox{16}{index'16}
\end{bytefield}
\caption{The layout of an index token.}
\label{fig:index_token}
\end{figure}

Like with lifetime tokens, the \textit{permissions} field is unmodified, but because there is no use for the permissions, every permission is set to zero.

The \textit{linear} bit is unaltered from the regular capability layout. Because index tokens should only be used to retrieve a stored capability once, they are always linear and are consumed in the retrieval operation.

The \textit{flags} field is unaltered from the regular capability layout.

The \textit{otype} field is always set to the identifier for index tokens: $2^{18} - 4$. This allows both the software and hardware to confirm whether a capability is an index token by checking its \textit{otype} field.

Index tokens do not use the fields related to bounds which means the $I_E$, \textit{T} and \textit{B} fields are unaltered from the regular capability layout and set to zero.

The previous \textit{a} field is split up into two different fields to hold the index in the borrow table and the lifetime id associated with the token. These two fields are 18 and 16 bits wide which means that the second word of an index token has 30 unassigned bits. The \textit{lifetime id} field indicates which lifetime id this index token corresponds to. A capability can only be retrieved when a dead lifetime token with the lifetime id in this field is present. The \textit{index} field indicates which capability should be retrieved when consuming the index token. The width of this field proportional to the size of the borrow table which is discussed in section \ref{sec:lcbt}.

\subsubsection{Memory Capabilities}
The layout of memory capabilities does not get changed with the introduction of borrowed capabilities. It remains the same layout that was introduced by linear capabilities in figure \ref{fig:linear_capability}. However, the introduction of borrowed capabilities does make a breaking change with regards to the \textit{otype} field.

Whereas previously almost the full otype range was reserved for code-data pairs, the introduction of borrowed capabilities splits this otype space in two, reserving half of it for use as lifetime id's. This results in three separate otype ranges. The first range, from 0 to $\frac{2^{18}}{2} - 1$ is used for lifetime id's. The first value of zero is not actually used as a valid lifetime id, but exists to represent the absence of a child or parent lifetime in lifetime tokens. The second range, from  $\frac{2^{18}}{2}$ to $2^{18} - 17$ is used for code-data pairs. The last range, from $2^{18} - 16$ to $2^{18} - 1$ is reserved for specific identifiers such as the lifetime token identifier and the index token identifier as discussed above. Note that borrowed memory capabilities can be identified by the presence of a lifetime id in their \textit{otype} field. Thanks to splitting the otype space in half, this can be accomplished by checking just the first bit of the otype field.

An alternative to this scheme would be to add an extra bit that determines whether the value held in the \textit{otype} field represents a lifetime id or a code-data pair. While this approach would keep compatibility with the existing CHERI architecture, it would also use the last of the leftover bits and complicate the interpretation of the \textit{otype} field as the interpretation would depend on a separate bit.

\subsection{Lifetime Counter \& Borrow Table}
\label{sec:lcbt}
Borrowed capabilities require a method to securely generate new lifetime id's. It should at all times be impossible to reuse an existing lifetime id as this would enable vulnerabilities such as using an old dead lifetime token to prove the death of a currently living lifetime. To generate new unique lifetimes we introduce an internal hardware register called the lifetime counter. This lifetime counter is initialized to one and incremented every time a new lifetime id gets created. The size of this lifetime counter corresponds to the size of the \textit{otype} field which is 18 bits. When the software tries to create a lifetime id that is out of the allowed range, a hardware exception is raised. This is further described in section \ref{sec:lifetimeinsts}.

Borrowed capabilities also require a table to store and retrieve capabilities that are being borrowed. From the perspective of the software, this is an opaque table that stores capabilities when they get borrowed and retrieves them when the software asks to. The only aspect of this table that is exposed to the software is the index of where a capability is stored. In microarchitecture, the table is most likely stored in system memory. The management of the table could be implemented by a privileged entity such as the kernel. In this scheme, instructions that need to access the table would trap to the operating system which would then store or retrieve table entries from system memory. Such a mechanism used to be present in older CHERI versions in the CCall instruction. %TODO reference this

The borrow table has the ability to
\begin{itemize}
\item Find free slots in the table.
\item Store a capability in a free slot and return the index of the slot in which it was stored.
\item Retrieve a capability in the slot with a specific index and clear the slot.
\end{itemize}

The size of the table was chosen to be $2^{16}$ entries, but is only really limited by the amount of space in index tokens. The layout of the index token that was discussed in the previous section contains 30 unassigned bits so the width of the \textit{index} field could theoretically grow to 46 bits, allowing for a table with a size of $2^{46}$ entries. However for performance and simplicity reasons, it seems desirable for the borrow table to be small. Thus a size of $2^{16}$ bits was chosen.

\subsection{Modified Instructions}

\label{sec:borrowmodinsts}
%TODO load and store instructions modified to need lifetime token, CSplitCap and CMergeCap, cseal to not allow lifetime ids, instructions that change offsets

\subsection{New Instructions}
\subsubsection{Lifetime Instructions}
\label{sec:lifetimeinsts}
\subsubsection{Borrow Instructions}
\subsubsection{Other Instructions}

\chapter{Assembler Implementation}
In order to test our implementation and execute sample programs, we need access to an assembler that supports our newly added instructions. We opted to extend CHERI's fork of the LLVM project that already has support for compiling and assembling code to CHERI supported architectures, including CHERI-RISC-V.

\section{LLVM Modifications}
The code snippet below shows all of the required modifications to LLVM to compile assembly programs that use the instructions added with borrowed capabilities. This snippet belongs in the \textit{RISCVInstrInfoXCheri.td} file, found in the \textit{llvm/lib/Target/RISCV/} directory, starting from the LLVM repository root.

%TODO formatting
\begin{verbatim}
let Predicates = [HasCheri] in {
def CGetLinear     : Cheri_r<0x14, "cgetlinear">;
let mayTrap = 1 in {
def CMakeLinear    : Cheri_r<0x13, "cmakelinear", GPCR>;
def CCreateToken   : Cheri_r<0x15, "ccreatetoken", GPCR>;
def CKillToken     : Cheri_r<0x16, "ckilltoken", GPCR>;
def CSplitLT       : Cheri_r<0x17, "csplitlt", GPCR>;
def CSplitCap      : Cheri_rr<0x1a, "csplitcap">;
def CMergeCap      : Cheri_rr<0x1b, "cmergecap", GPCR, GPCR>;
def CUnlockToken   : Cheri_rr<0x15, "cunlocktoken", GPCR, GPCR>;
def CMergeLT       : Cheri_rr<0x16, "cmergelt", GPCR, GPCR>;
def CBorrowImmut   : Cheri_rr<0x17, "cborrowimmut", GPCR, GPCR>;
def CBorrowMut     : Cheri_rr<0x18, "cborrowmut", GPCR, GPCR>;
def CRetrieveIndex : Cheri_rr<0x19, "cretrieveindex", GPCR, GPCR>;
}
}

let Predicates = [HasCheri, IsRV64] in
def LL_CAP_128 : CheriLoad_r<0b00111, "ll.cap", GPCR, GPCRMemAtomic>;

let Predicates = [HasCheri, IsRV64] in
def SL_CAP_128 : CheriStore_r<0b01101, "sl.cap", GPCR, GPCRMemAtomic>;
\end{verbatim}

The code snippet defines all of the new instructions and specifies how they need to be encoded. Most of the instructions make use of the \textit{Cheri\_r} or \textit{Cheri\_rr} templates, discussed in section \ref{} %TODO reference background
The first argument to these templates, the funct5 or funct7 arguments, express how the mnemonic stated in the second argument should be encoded to bits. The funct5 or funct7 arguments correspond to the encoding as shown in section \ref{sec:sailencoding}. The remainder of the arguments are optional and used to override the defaults in the template. Passing \textit{GPCR} to the template expresses that the corresponding register should be interpreted as a capability register.

The \textit{ll.cap} and \textit{sl.cap} instructions that stand for the linear load and store instructions use a different template, corresponding to their purpose, but follow the same principles as above. The first argument expresses the encoding, the second argument is the mnemonic to be used and the third and fourth arguments define the interpretation of operands.

The use of hexadecimal notation for the first set of instructions and binary notation for the load and store instructions follows the style of the already existing code in the file.

\section{Compiling Assembly}
\begin{verbatim}
clang -nostdlib --target=riscv64-unknown-elf-64 -march=rv64gcxcheri
-mno-relax -Ttext 0x80000000 -o out.elf asm.s
\end{verbatim}
This command is the command used to assemble an assembly file called \textit{asm.s} to an executable ELF binary called \textit{out.elf}. The command uses the clang executable that was produced by compiling LLVM with the modifications described above. The flags tell clang to produce a CHERI-RISC-V ELF file without linking against the standard library with linker relaxation enabled. The \textit{-Ttext 0x80000000} tells the emulator to load the generated code in memory starting at address \textit{0x80000000}. This necessary because the emulator does not consider smaller addresses as ram and will refuse to execute the program.

\section{Considerations}
LLVM instructions are usually connected to certain directives with the purpose of improving the quality of the code generated by the compiler. One such directive can be seen in the snippet with the \textit{mayTrap} directive. Since we do not need a compiler for the purposes of this thesis, we do not concern ourselves with the efficiency or even correctness of the usage of these directives. These directives are ignored when using the assembler, like we do in this thesis.

\chapter{Evaluation}

\section{Assembly Explanation}
In this section we explain the assembly code that is required to set up the program and is common to all of the examples we will show in this chapter.
To save space and to focus on the essential part of the assembly, we will not repeat the code that is given here.

The assembly snippet starts off by defining the \textit{\_start} label which is read by the emulator and used to determine where the emulator should start executing code.
The code at the \textit{\_start} label loads the address of the \textit{\_trap\_vector} label in the \textit{t0} register and then sets that address as the address to trap to when an instruction encounters an exception with the \textit{csrw} instruction.
The following two lines do something similar but set the return address that is returned to when executing the \textit{mret} instruction instead.
Then the \textit{mret} causes the emulator to enter user mode and jump to the \textit{\_user} label.
The \textit{\_user} label is where the code that we give throughout this chapter will be placed when executing the examples.
The code at the \textit{\_trap\_vector} label loads the value 3 in a register that will be used to pass the return code of the program to the emulator and then jumps to the \textit{\_exit} label.
The code at the \textit{\_exit} label writes the value in \textit{gp} to the \textit{tohost} address which is considered communication with the emulator.
The emulator considers any value with the least significant bit set to 1 as a message to stop the emulation with a return code defined by the number in the other bits of the value.
A return code of zero means that the program exited successfully while any other value signifies an error code that corresponds to a failure in the program.
In the section of code at \textit{\_trap\_vector} we set the value to 3 which corresponds to a failure with error code 1.
This happens when some instruction in our program raised an exception.
In the other assembly snippets we will give throughout this chapter, we will set the value of \textit{gp} to 1 to signify success.
The final line of the assembly defines the \textit{tohost} symbol and address so that the emulator knows which address it should be listening to.
\begin{verbatim}
        .global _start
_start:
        la t0,  _trap_vector
        csrw	mtvec,t0

        la      t0, _user
        csrw    mepc, t0
        mret

.align 4
_trap_vector:
        li      gp,3
        j       _exit

_exit:
        auipc   t5, 0x1
        sw      gp, tohost, t5
        j       _exit

_user:
        ...

.align 6; .global tohost; tohost: .dword 0;
\end{verbatim}
%TODO refer to the github where this was inspired from

\section{Instruction Testing}

\section{Comparison to Rust}

\section{Performance}

\chapter{Discussion}
\label{chap:discussion}
In this chapter we discuss some considerations and shortcomings of our design of borrowed capabilities.

\section{Nested references}
In section \ref{sec:rust_nested} we explained Rust's rules around load references through a borrow.

In listing \ref{code:nested_borrow} we demonstrated that a mutable reference has to be borrowed when loaded through a borrow.
This is not supported in our design for borrowed capabilities because the borrow operations work only on registers, not on memory.
A solution for this would involve a new instruction that loads the mutable capability, stores it in the borrow table, places a borrowed capability in the output register with a lifetime id that matches a provided lifetime token and writes an index token to memory.
This instruction that is a fusion of a load and borrow operation would be very complex and it might not be realistic to implement this in microarchitecture.
Additionally, any reference that is loaded in this way is actually a reborrow of the nested reference and thus needs to have a shorter lifetime than the nested reference itself, regardless of whether it is a mutable or immutable reference.
In our design this means that the lifetime token provided to the load-borrow instruction needs to be a child of the lifetime of the nested reference, adding further complexity to the instruction.
Another issue is that such an instruction would give issues with loading references that already use their \textit{otype} field for another purpose because then the \textit{otype} would not be available for a lifetime id.

Another issue is that immutable references can not be used to load mutable references as explained in \ref{code:nested_immutmut}.
Instead, the loaded references are made immutable which in turn renders them unable to load mutable references.
Simply unsetting the write permissions on a capability that is loaded through an immutable borrowed capability is not sufficient as this loaded capability would be able load mutable capability itself, losing the recursive property.
A possible solution would be to use the experimental CHERI \textit{Permit\_Recursive\_Mutable\_Load} permission that removes store permissions and this permission itself on any capability that is loaded through a capability that does not have this permission set.
%TODO ref

\section{Semantic Lifetimes}
In section \ref{sec:semantic_lifetimes} we explained how Rust detects the last usage of a borrow in order to make the lifetime as short as possible.
This is a way to make Rust code more flexible and readable, but has its limitations because the lifetimes are still determined statically.
In essence, borrowed capabilities do something similar by making lifetimes dynamic.
With dynamic lifetimes, the specific lifetimes of certain borrows do not have to be known at compile time which makes them more flexible than Rust's semantic lifetimes.
This would create opportunities to make Rust code more flexible as borrowed capabilities would enforce Rust's guarantees dynamically.

\section{Performance}
As mentioned in chapter \ref{chap:evaluation} we do not evaluate our design on performance, because this is hard without actual hardware.
This is a problem that is not unique to our design, but has actually been an issue for CHERI research for a while.
While CHERI prototypes on FPGA's have been used to get an idea about performance, these prototypes are still far from an actual processor that is comparable to normal modern processors.
One interesting development in this field is the Morello project which is a collaboration between the CHERI developers and ARM to create an actual processor that supports CHERI.
This CHERI processor would give a much better idea about the performance impact of CHERI in general.
%TODO ref

As for borrowed capabilities, one performance measure that we could have used to evaluate our design is a sort of microbenchmark that counts the amount of additional instructions required for a program that uses borrowed capabilities.
This was not attempted due to a lack of time and would be complicated because it requires a non-trivial assembly program with a borrowed capability version and a baseline version for comparison.
This might also not give a very accurate picture as some instructions are more expensive than others.
Any operation that requires an access to the borrow table would probably require accessing memory which is much more expensive than a simple move instruction for example.
This problem is further exacerbated due to the fact that the cost of accessing memory varies greatly depending on whether the processor cache is used or not.
%TODO moving lifetime tokens around to get them into C31 is probably expensive as well

\section{Borrowed Capabilities for Revocation}
In this section we discuss borrowed capabilities as a concept that is not tied to Rust's semantics, but as a way to revoke capabilities on CHERI.
Through this lens, borrowed capabilities allow software to create scopes in the form of lifetime tokens and allow capabilities to be bound to these scopes.
The revocation problem then shifts to the revocation of lifetimes as opposed to the revocation of individual capabilities.
Revoking a lifetime works in a similar manner as revoking linear capabilities, but avoids the restrictive copy prohibition on capabilities that linear references have.
It also avoids the restriction on storing local capabilities in non-write-local memory while still being able to do fine grained revocation of capabilities.

Borrowed capabilities follow the ``aliasing XOR mutation'' principle, which means that they offer revocation in a setting where there is mutual distrust between two parties.
A callee that is given a mutable borrowed capability and a matching lifetime token can be certain that its caller cannot access the resource that the capability points to because the caller's version of the capability is stored away in the borrow table and can only be retrieved by killing the lifetime token that the callee holds.
In the other direction, a caller can be sure that a callee does not have access to a resource anymore when the caller holds the dead lifetime token.
Something to note is that mutual distrust requires that only linear capabilities can be borrowed since a callee has to be sure no copies of the capability exist.
``Aliasing XOR mutation'' also implies the need for two borrow operations because mutable borrows need to be linear.
This linearity requirement means that it is not possible to create an immutable borrow from a mutable borrow by removing write permissions as there is no way to remove the linearity on the mutable borrow.

The design of borrowed capabilities for revocation can be simplified significantly if we don't require ``aliasing XOR mutation'' or mutual distrust.
In this case a caller just wants to be sure that a callee cannot access a resource anymore after it has returned.
This removes the requirement for a borrow table as the callee can just hold on to the capability that is being borrowed.
It would allow for non linear capabilities to be borrowed since the callee wouldn't care about the caller holding copies.
Another interesting simplification is that mutably borrowed capabilities would not need to be linear which leads to the possibility of having only one borrow operation.
This borrow operation would result in a borrowed capability with the same permissions as the input capability.
If the caller wanted to restrict the callee's write access, it could still do that by removing the permissions on the borrowed capability.

\section{Usage of fractions}

\chapter{Conclusion}
\label{cha:conclusion}
In this thesis we presented \textit{borrowed capabilities}, a new type of capability for CHERI that is designed to mirror the semantics of the Rust programming language.
We introduced the design for borrowed capabilities, revolving around binding capabilities to lifetime tokens and making these bound capabilities only dereferenceable when that lifetime token is present.
The design provides support for the ``aliasing XOR mutation'' principle by locking away capabilities in a borrow table and having both mutable and immutable borrow operations.
Concurrent use cases were taken into account by allowing lifetime tokens to split into multiple lifetime fractions that can be used by multiple threads at the same time.
We made a Sail implementation of both linear capabilities and borrowed capabilities and discussed the various design decisions relating to this implementation.
We extended the LLVM compiler project to be able to assemble programs using the instructions that were introduced in the design.
Finally we wrote sample assembly programs that mirror specific Rust programs to demonstrate how borrowed capabilities offer semantics that are similar to Rust's ownership and borrowing system and reasoned about the costs of the design of borrowed capbilities.
The assembly programs can be assembled by the LLVM assembler and run in the emulator that is produced by Sail.
From the assembly programs we can conclude that borrowed capabilities do indeed have semantics and guarantees that are similar to those that Rust provides.

There are some outstanding problems with borrowed capabilities that prevent them from being used as the target of a secure Rust compiler such as the behavior related to nested references.
Nevertheless, we think borrowed capabilities are a significant step towards supporting Rust's ownership and borrowing system at the assembly level.
While fitting the design of borrowed capabilities into CHERI is fairly simple and does not require expensive changes to CHERI, actually implementing borrowed capabilities in hardware could have significant extra costs.
The main limitations of borrowed capabilities such as lifetime id exhaustion may be mitigated or worked around.

We also considered borrowed capabilities as a new form of revocation in CHERI, supporting a model of mutual distrust between two parties.
While this thesis did not focus on the use of borrowed capabilities as a general form of revocation, we do think they are promising in this regard and this is an interesting direction to explore in future work.


%%% Local Variables: 
%%% mode: latex
%%% TeX-master: "thesis"
%%% End: 


% If you have appendices:
\appendixpage*          % if wanted
\appendix
\chapter{Sail Documentation}
\label{app:saildoc}
\includepdf[pages=-]{sailcode.pdf}

%%% Local Variables: 
%%% mode: latex
%%% TeX-master: "thesis"
%%% End: 

\chapter{Short Paper}
\label{app:paper}
\includepdf[pages=-]{paper.pdf}

%%% Local Variables: 
%%% mode: latex
%%% TeX-master: "thesis"
%%% End: 


\backmatter
% The bibliography comes after the appendices.
% You can replace the standard "abbrv" bibliography style by another one.
\bibliographystyle{abbrv}
\bibliography{references}

\end{document}

%%% Local Variables: 
%%% mode: latex
%%% TeX-master: t
%%% End: 
